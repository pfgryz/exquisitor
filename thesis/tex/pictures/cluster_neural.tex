{
% ===== BEGIN =====
% ----- -----
% COLORS
% ----- -----
\definecolor{Green}{HTML}{1dd1a1}
\definecolor{Blue}{HTML}{54a0ff}
\definecolor{Yellow}{HTML}{feca57}
\definecolor{Purple}{HTML}{5f27cd}
\definecolor{Grey}{HTML}{576574}
\definecolor{Red}{HTML}{ff6b6b}
\definecolor{Pink}{HTML}{ff9ff3}
\definecolor{Background}{HTML}{c8d6e5}

% ----- -----
% ELEMENTS
% ----- -----
\tikzstyle{Circle} = [circle, minimum size=1cm, line width=2pt, draw=black]
\tikzstyle{Box} = [rectangle, minimum width=10cm, minimum height=1.5cm, line width=2pt, text centered, inner sep=10pt, draw=black]
\tikzstyle{Arrow} = [very thick, -Triangle]
\tikzstyle{Arrow:Text} = [pos=0.5, right, font=\footnotesize]

% ----- -----
% PICTURE
% ----- -----
\begin{tikzpicture}[node distance=3cm]
    \node (input) [Circle] { Wejście };
    \node (embed) [Box, below of=input, align=center, draw=Red] { 1. Otrzymanie wektorów cech \\ \textbf{za pomocą sieci neuronowej} };
    \node (distance) [Box, below of=embed, align=center] { 2. Obliczenie niepodobieństwa \\ między wektorami cech };
    \node (matrix) [Box, below of=distance] { 3. Stworzenie macierzy niepodoobieństwa };
    \node (cluster) [Box, below of=matrix, align=center] { 4. Grupowanie sekwencji \\ \textbf{za pomocą algorytmu k-medoidów}};
    \node (output) [Circle, below of=cluster] { Wyjście };

    \draw [Arrow] (input) -- (embed) node [Arrow:Text] {Zbiór sekwencji DNA};
    \draw [Arrow] (embed) -- (distance) node [Arrow:Text] {Wektory cech};
    \draw [Arrow] (distance) -- (matrix) node [Arrow:Text] {Niepodobieństwo sekwencji};
    \draw [Arrow] (matrix) -- (cluster) node [Arrow:Text] {Macierz niepodobieństwa};
    \draw [Arrow] (cluster) -- (output) node [Arrow:Text] {Grupy};
\end{tikzpicture}

% ===== END =====
}