\cleardoublepage{}

\abstract{}
Celem niniejszej pracy dyplomowej jest stworzenie aplikacji służącej do przeprowadzania analiz materiału genetycznego DNA pochodzącego od różnych organizmów, pobranego ze środowiska. Aplikacja wykorzystuje bazę danych sekwencji genetycznych oraz wprowadza usprawnienie w procesie analizy dzięki zastosowaniu różnych metod, w tym uczenia maszynowego.
Wykorzystano klasyfikację taksonomiczną sekwencji genetycznych do przeprowadzania analizy materiału genetycznego. Optymalizacja procesu została osiągnięta poprzez grupowanie sekwencji genetycznych oraz wybór reprezentantów sekwencji, które posłużyły do klasyfikacji taksonomicznej. W ramach pracy zaimplementowano dwie metody klasyczne: wykorzystującą wyrównania sekwencji oraz analizę opartą na $k$-merach. Opracowano również autorską metodę opartą na uczeniu maszynowym wykorzystującą sztuczne sieci neuronowe wraz z uczeniem kontrastowym. Stworzono aplikację konsolową, przeprowadzającą proces analizy, oraz aplikację internetową umożliwiającą użytkownikowi zlecanie analiz materiału genetycznego. Implementację stworzono w języku Rust.
Przeprowadzono eksperymenty jakościowe, w których porównano zaimplementowane metody względem pełnej klasyfikacji taksonomicznej wszystkich sekwencji, oraz eksperymenty wydajnościowe, oceniające czas wykonania całego procesu. Wyniki wykazały, że opracowana metoda autorska cechowała się najwyższą jakością klasyfikacji oraz najszybszym czasem działania w porównaniu z pozostałymi metodami.

\keywords{}
sekwencje DNA, klasyfikacja taksonomiczna, metagenomika, sztuczne sieci neuronowe, uczenie kontrastowe
