\cleardoublepage

\section{Wprowadzenie}

    Odkrycie DNA zapoczątkowało nową erę w biologii oraz medycynie, umożliwiając badania nad molekularnymi podstawami życia oraz precyzyjną diagnostykę wielu chorób\cite{Louie:2000}. Pierwsze kroki na tej drodze uczynił w 1869 roku Friedrich Miescher, który po raz pierwszy wyizolował z jądra komórkowego substancję nazwaną przez niego ,,nukleiną''\cite{Dahm:2005}, później zidentyfikowaną jako kwas deoksyrybonukleinowy. W latach 1895 - 1901 Albrecht Kossel wyizolował oraz nazwał cztery podstawowe zasady azotowe budujące kwas deoksyrybonukleinowy — adeninę, tyminę, cytozynę oraz guaninę\cite{Kossel:1893}. Przełomowe znaczenie DNA w przenoszeniu informacji genetycznej wykazano jednak dopiero w 1944 roku dzięki eksperymentowi Avery'ego-MacLeoda-McCarty'ego\cite{Avery:1944}, w którym udowodnili, że to DNA, a nie białka są nośnikiem informacji. W 1951 Erwin Chargaff odkrył prawidłowość, że ilość adeniny w DNA jest porównywalna do ilości tyminy, a ilość guaniny do ilości cytozyny\cite{Chargaff:1952}. Prawidłość ta, nazwa później na cześć odkrywcy zasadami Chargaffa, stała się jedną z kluczowych przesłanek do odkrycia struktury podwójnej helisy DNA przez Watsona i Cricka w 1953 roku\cite{Watson:1953}.

    W 1970 Francis Crick sformułował centralny dogmat biologii molekularnej\cite{Crick:1970}, który głosi, że informacja genetyczna przepływa z DNA do RNA, a następnie do białek tworząc fundament współczesnej biologii molekularnej. Dalsze przełomowe odkrycia metod sekwencjonowania, opracowane w 1976 roku przez Allana Maxima oraz Waltera Gilberta\cite{Maxam:1977}, a także w 1977 roku przez Fredericka Sangera\cite{Sanger:1977}, pozwoliły na odczytywanie sekwencji DNA dowolnych organizmów, otwierając nowe możliwości badań nad DNA oraz przepływem informacji genetycznej. W tym samym czasie zespołowi Fredericka Sangera za pomocą nowej metody udało się po raz pierwszy zsekwencjonować w całości materiał genetyczny wirusa DNA — bakteriofaga $\phi{}$X174\cite{Sanger:1977_2}. Odkrycie to uwypukliło ograniczenia tradycyjnych metod analizy i wykazało konieczność zastosowania komputerów do przetwarzania sekwencji DNA\cite{Staden:1979}.

    W miarę postępu badań nad sekwencjami DNA, zespoły badawcze rozpoczęły gromadzenie sekwencji, w celu dalszych analiz i porównań. Pod koniec lat 70. XX wieku pojawiła się potrzeba stworzenia bazy danych, która umożliwiałaby gromadzenie sekwencji w celu ich późniejszego wykorzystania oraz redukcji kosztów badań. Odpowiedzią na to zapotrzebowanie było sfinansowanie bazy danych GenBank w 1982 roku w Stanach Zjednoczonych\cite{Bilofsky:1986} oraz europejskiej bazy danych EBML Data Library w 1980 roku\cite{Higgins:1992}. W 1983 roku Johnowi Wilburowi oraz Davidowi Lipmanowi udało się opracować algorytm wyszukiwania sekwencji podobnych w bazach danych\cite{Wilbur:1983}, co pozwoliło na szybkie porównywanie sekwencji genetycznych. W 1990 roku wprowadzono narzędzie BLAST\cite{Altschul:1990}, które przyśpieszyło ten proces.

    Pierwszy pomysł analizy materiału genetycznego wielu organizmów jednocześnie bez konieczności ich hodowli został zaproponowany w 1998 roku przez Jo Handlesmana i jego zespół\cite{Handelsman:1998}. Zespół badał organizmy obecne w glebie. Wykorzystanie informacji o materiale genetycznym znalezionym w środowisku, w połączeniu z bazami danych sekwencji, umożliwia  identyfikację organizmów obecnych w badanej próbce. Proces analizy dużej liczby sekwencji DNA oraz ich porównywanie z bazami danych wymaga jednak znacznych zasobów obliczeniowych.

    Wraz z rozwojem metod sekwencjonowania nowej generacji\cite{Reinartz:2002}, koszty sekwencjonowania znacząco spadły, a liczba analizowanych sekwencji DNA znacznie wzrosła\cite{Muir:2016}. Dalsze gromadzenie danych sekwencyjnych oraz wzrost przepustowości technologii sekwencjonowania prowadzą do zwiększonego zapotrzebowania na zasoby obliczeniowe.

    % ===== ===== ===== =====
    % CEL PRACY
    % ===== ===== ===== ===== 
    \subsection{Cel pracy dyplomowej}

        Celem niniejszej pracy jest stworzenie aplikacji umożliwiającej użytkownikowi analizę materiału genetycznego DNA z wielu organizmów, pobranego ze środowiska. Aplikacja będzie wykorzystywać bazy danych sekwencji genetycznych i ma na celu usprawnienie procesu analizy danych oraz agregacji wyników w formie czytelnej dla człowieka. Niezbędne będzie opracowanie nowych algorytmów, szczególnie w obszarze grupowania sekwencji oraz wyboru reprezentantów, w tym z wykorzystaniem modeli uczenia maszynowego.

    % ===== ===== ===== =====
    % ZAKRES PRACY
    % ===== ===== ===== ===== 
        \subsection {Zakres pracy}

        \todo{listy, wyliczenia w pracy, spójnie: np. po dwukropku, z małej, na koniec wyliczenia średnik, na koniec całości kropka.}

        Zakres pracy obejmuje:
        \begin{itemize}
            \item {
                projekt i implementację aplikacji do analizy materiału genetycznego;
            }
            \item {
                klienta przeglądarkowego pozwalającego na zlecanie nowych analiz;
            }
            \item {
                funkcjonalności pozwalające na wczytanie sekwencji w formacie FASTA i FASTQ oraz wyszukiwanie w bazie danych NCBI BLAST;
            }
            \item {
                implementację własnej metody porównywania sekwencji DNA wykorzystującej sztuczne sieci neuronowe;
            }
            \item {
                implementację metod klasycznych do porównywania sekwencji DNA;
            }
            \item {
                eksperymenty wydajnościowe oraz jakościowe klasycznych metod, oraz metody własnej.
            }
        \end{itemize}

        Praca nie obejmuje
        \begin{itemize}
            \item {
                Wdrożenia aplikacji w środowisku produkcyjnym.
            }
            \item {
                Wykorzystania innych baz sekwencji genetycznych.
            }
            \item {
                Wersji mobilnej klienta przeglądarkowego.
            }
        \end{itemize}

    % ===== ===== ===== =====
    % STRUKTURA PRACY
    % ===== ===== ===== ===== 
    \subsection {Struktura pracy}

    Niniejsza praca składa się z następujących rozdziałów:
    \todo{po dwukropku z małej litery, więc może lepiej to zmienić zdanie powyżej na: 'Praca składa się z pięciu rozdziałów.'}

        \begin{itemize}
            \item {
                Rozdział 2. skupia się na przeglądzie literatury, analizie istniejących badań oraz przedstawieniu obecnych rozwiązań. Dodatkowo zawiera on podstawowe pojęcia i definicje.
            }
            \item {
                Rozdział 3. zawiera szczegółowy projekt oraz opis implementacji zaproponowanego rozwiązania. Zawiera on opis wymagań funkcjonalnych, niefunkcjonalnych, model architektury rozwiązania oraz opis zaprojektowanego systemu. Przedstawiono w nim strukturę potoku przetwarzania danych, podział na poszczególne komponenty systemu, opisano zastosowane metody oraz wykorzystane technologie. Ponadto zawiera on opis interfejsu użytkownika wraz z instrukcją użytkownika.    
            }
            \item {
                Rozdział 4. zawiera opis przeprowadzonych eksperymentów, w tym opis środowiska eksperymentalnego, zastosowaną miarę jakości, opis wykorzystanego zbioru danych oraz procedurę przeprowadzania eksperymentów. Rozdział kończy się prezentacją wyników oraz ich analizą.
            }
            \item {
                Rozdział 5. obejmuje podsumowanie, w którym omówione zostały główne cele pracy, osiągnięte wyniki oraz kluczowe osiągnięcia. Zawiera także omówienie napotkanych ograniczeń oraz wskazanie możliwości dalszego rozwoju systemu.
            }
        \end{itemize}
