\clearpage

\todo{rozdzialy zaczynamy 'cleardoublepage'}
\todo{,,polskie cytownania''}
\todo{usunąć symulowane cytowania, tzn wpisy typu [ X ] }

\section{Wprowadzenie}

Odkrycie DNA zapoczątkowało nową erę w biologii oraz medycynie, umożliwiając badania
nad molekularnymi podstawami życia oraz precyzyjną 
diagnostykę wielu chorób \cite{Louie:2000}. Pierwsze kroki na tej drodze uczynił w 1869 roku 
Friedrich Miescher, który po raz pierwszy wyizolował z jądra komórkowego substancję 
nazwaną przez niego "nukleiną" \cite{Dahm:2005}, później zidentyfikowaną jako kwas deoksyrybonukleinowy. 
W latach 1895 - 1901 Albrecht Kossel wyizolował oraz nazwał cztery podstawowe zasady 
azotowe budujące kwas deoksyrybonukleinowy - adeninę, tyminę, cytozynę oraz guaninę \cite{Kossel:1893}.
Przełomowe znaczenie DNA w przenoszeniu informacji genetycznej wykazano jednak dopiero 
w 1944 roku dzięki eksperymentowi Avery'ego-MacLeoda-McCarty'ego \cite{Avery:1944}, w którym udowodnili, 
że to DNA, a nie białka są nośnikiem informacji. 
W 1951 roku Erwin Chargaff odkrył prawidłowość, że ilość adeniny w DNA jest porównywalna
ilości tyminy, a ilość guaniny ilości cytozyny \cite{Chargaff:1952}. Prawidłowość ta, nazwana później zasadami
Chargaffa, stała się jedną z kluczowych przesłanek do odkrycia struktury podwójnej helisy 
DNA przez Watsona i Cricka w 1953 roku \cite{Watson:1953}.

W 1970 Francis Crick sformułował centralny dogmat biologii molekularnej \cite{Crick:1970}, który głosi,
że informacja genetyczna przepływa z DNA do RNA, a następnie do białek, tworząc 
fundament współczesnej biologii molekularnej. Dalsze przełomowe odkrycia metod sekwencjonowania,
opracowane w 1976 roku przez Allana Maxima oraz Waltera Gilberta \cite{Maxam:1977}, a także w 1977 roku 
przez Fredericka Sangera \cite{Sanger:1977}, pozwoliły na odczytywanie sekwencji DNA dowolnych organizmów, 
otwierając nowe możliwości badań nad DNA oraz przepływem informacji genetycznej. 
W tym samym czasie zespołowi Fredericka Sangera za pomocą nowej metody udało się 
po raz pierwszy zsekwencjonować w całości materiał genetyczny wirusa DNA - 
bakteriofaga  X174 \cite{Sanger:1977_2}. Odkrycie te uwypukliło ograniczenia tradycyjnych metod  % TODO: brakuje PHI przy X
analizy i wykazało konieczność zastosowania komputerów do przetwarzania sekwencji DNA. \cite{Staden:197 9}

\todo{powyższe cytowanie błędne, poniżej dużo 'pseudocytowań'}

W miarę postępu badań nad sekwencjami DNA, zespoły badawcze zaczęły gromadzić sekwencje 
w celu dalszych analiz i porównań. Pod koniec lat 70. XX wieku pojawiła się potrzeba 
stworzenia bazy danych, która umożliwiłaby gromadzenie sekwencji w celu ich późniejszego 
wykorzystania oraz redukcji kosztów badań. Odpowiedzią na to zapotrzebowanie było 
sfinansowanie bazy danych GenBank w 1982 w Stanach Zjednoczonych[13] oraz 
europejskiej bazy danych EBML Data Library w 1980 roku [14]. W 1983 roku 
Johnowi Wilburowi oraz Davidowi Lipmanowi udało się opracować algorytm wyszukiwania
podobnych sekwencji w bazach danych [15], co pozwoliło na szybkie porównywanie sekwencji 
genetycznych. W 1990 roku wprowadzono narzędzie BLAST [16], które przyśpieszyło ten 
proces.

Pierwszy pomysł na analizę materiału genetycznego wielu organizmów jednocześnie,
konieczności ich hodowli, został zaproponowany w 1998 roku przez Jo Handlesmana i 
jego zespół [17]. Zespół badał organizmy obecne w glebie.
Wykorzystanie informacji o materiale genetycznym znalezionym w 
środowisku, w połączeniu z bazami danych sekwencji, umożliwia identyfikację organizmów 
obecnych w badanej próbce [??]. Proces analizy dużej liczby sekwencji DNA oraz ich 
porównywania z bazami danych wymaga jednak znacznych zasobów obliczeniowych.

Wraz z rozwojem metod sekwencjonowania nowej generacji [18], koszty sekwencjonowania 
znacząco spadły, a liczba analizowanych sekwencji DNA znacznie wzrosła [19]. Dalsze 
gromadzenie danych sekwencyjnych oraz wzrost przepustowości technologii sekwencjonowania 
prowadzą do zwiększonego zapotrzebowania na zasoby obliczeniowe.


---- REFERENCES 
1. XYZ
2. https://pmc.ncbi.nlm.nih.gov/articles/PMC80298/
3. https://www.sciencedirect.com/science/article/pii/S0012160604008231?via%3Dihub
4. Kossel, A. and Neumann, A.: Ueber das Thymin ein Spaltungsprodukt der Nucleinsaure. Ber. Deut. chem. Ges., 1893, 26, 2753.
5. https://pubmed.ncbi.nlm.nih.gov/19871359/c
6. https://www.sciencedirect.com/science/article/pii/S0021925819508845?via%3Dihub
7. https://www.nature.com/articles/171737a0

8. https://pubmed.ncbi.nlm.nih.gov/4913914/
9. https://pubmed.ncbi.nlm.nih.gov/265521/
10. https://pubmed.ncbi.nlm.nih.gov/271968/
11. https://pubmed.ncbi.nlm.nih.gov/870828/
12. https://pmc.ncbi.nlm.nih.gov/articles/PMC327874/

13. https://doi.org/10.1093/nar/14.1.1
14. https://pmc.ncbi.nlm.nih.gov/articles/PMC333983/
15. https://pubmed.ncbi.nlm.nih.gov/6572363/
16. https://pubmed.ncbi.nlm.nih.gov/2231712/

17. https://www.sciencedirect.com/science/article/pii/S1074552198901089?via%3Dihub 
18. https://pubmed.ncbi.nlm.nih.gov/15251069/
19. https://pmc.ncbi.nlm.nih.gov/articles/PMC4806511/


    % ===== ===== ===== =====
    % CEL PRACY
    % ===== ===== ===== ===== 
    \subsection {
        Cel pracy dyplomowej
    }

    Celem niniejszje pracy jest stworzenie aplikacji umożliwiającej użytkownikowi
    analizę materiału genetycznego DNA z wielu ogranizmów, pobranego z środowiska.
    Będą wykorzystane bazy danych sekwencji.
    Materiał genetyczny pochodzi z wielu organizmów.
    Aplikacja ma usprawnić proces przeprowadzania analiz oraz proces 
    agregacji końcowych wyników do postaci czytelnej przez człowieka.
    Wymagać to będzie nowych algorytmów, w szczególności grupowania i wybierania reprezentantów,
    w tym wykorzystanie modeli uczenia maszynowego.

    % ===== ===== ===== =====
    % ZAKRES PRACY
    % ===== ===== ===== ===== 
    \subsection {
        Zakres pracy
    }

    \todo{kropki na końcach zdań}

        Zakres pracy obejmuje:
        \begin{itemize}
            \item Projekt i implementację aplikacji do analizy materiału genetycznego.
            \item Klienta przeglądarkowego pozwalającego na zlecanie nowych analiz.
            \item {
                Funkcjonalności pozwalające na wczytanie sekwencji DNA w formacie FASTA oraz FASTQ
                oraz wyszukiwanie w bazie danych NCBI BLAST
            }
            \item Implementację własnej metody porównywania sekwencji \todo{wykorzystującej dokończyć...}
            \item Eksperymenty wydajnościowe oraz jakościowe klasycznych metod oraz metody własnej
        \end{itemize}

        Praca nie obejmuje:
        \begin{itemize}
            \item Wdrożenia aplikacji w środowisku produkcyjnym
            \item Wykorzystania innych baz sekwencji
            \item Wersji mobilnej klienta przeglądarkowego
        \end{itemize}

    % ===== ===== ===== =====
    % STRUKTURA PRACY
    % ===== ===== ===== ===== 
    \subsection {
        Struktura pracy
    }

        Niniejsza praca składa się z następujących rozdziałów:

        \begin{itemize}
            \item {
                Rozdział 2. skupia ię na przeglądzie literatury, analizie istniejących
                badań oraz przedstawieniu obecnych rozwiązań.
            }
            \item {
                Rozdział 3. zawiera wstęp teoretyczny, objaśniający definicję oraz 
                operacje na sekwencjach oraz wykorzystywane algorytmy.
            }
        \end{itemize}













% ====================================================================================


**TODO** [ok. 3 strony]

\begin{enumerate}
    \item Krótki opis problemu [akapit]
    \item Cel pracy (co zamierzam zrobić) [1 akapit]
    \item Zakres pracy (jak zamierzam zrobić) [1 akapit]
    \item Układ pracy (co w którym rozdziale) [1 akapit]
\end{enumerate}



