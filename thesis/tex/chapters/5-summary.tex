\cleardoublepage
\section{Podsumowanie}

    % ===== ===== ===== =====
    % GŁÓWNE CELE PRACY
    % ===== ===== ===== =====
    \subsection{Główne cele pracy}
        \todo{
            \begin{enumerate}
                \item {krótkie przepomnienie celu pracy oraz eksperymentów,}
                \item {podsumowanie problemów, które zostały rozwiązane}
            \end{enumerate}
        }

        \todo {
            DO NAPISANIA \\
            \textbf{Główne cele}
            \begin{itemize}
                \item {CEL 1: Stworzenie aplikacji do przeprowadzania analiz}
                \item {CEL 2: Usprawnienie procesu analizowania sekwencji}
                \item {CEL 2.5: Implementacja trzech metod grupowania dwóch klasycznych i jednej własnej}
                \item {EXP 1: Eksperyment wydajnościowy - czas wykonania klasyfikacji}
                \item {EXP 2: Eksperyment jakościowy - jakość klasyfikacji}
            \end{itemize}

            \textbf{Rozwiązane problemy}
            \begin{itemize}
                \item {ROZWIĄZANIE 1: Stworzono aplikację, która pozwala na zlecanie analiz przez przeglądarkę}
                \item {ROZWIĄZANIE 2: Zastosowanie grupowanie}
                \item {ROZWIĄZANIE 2.5: Zaimplementowano metodę NW, $k$-mer oraz stworzono model uczenia maszynowego}
            \end{itemize}
        }

    % ===== ===== ===== =====
    % WYNIKI I OSIĄGNIĘCIA
    % ===== ===== ===== =====
    \subsection{Wyniki i osiągnięcia}
        \todo{
            \begin{enumerate}
                \item {najważniejsze cechy wytworzonego rozwiązania,}
                \item {wnioski z eksperymentów}
            \end{enumerate}
        }
        \todo{
            DO NAPISANIA \\\
            \textbf{Cechy}
            \begin{itemize}
                \item {CECHA: Stworzono rozwiązanie, które pozwala na łatwy dostęp do użytkownika - aplikacja internetowa}
                \item {CECHA: Jest aplikacja konsolowa do pracy wsadowej}
                \item {CECHA: Rozwiązanie zaimplementowane w Rust zapewnia brak wycieków pamięci}
            \end{itemize}

            \textbf{Wnioski}
            \begin{itemize} 
                \item {WNIOSEK: Nowa metoda okazała się dość szybka, porównywalna z kmerami i szybsza od klasycznej klasyfikacji}
                \item {WNIOSEK: Nowa metoda ma wyższą jakość od pozostałych metod klasycnzych}
            \end{itemize}
        }

    % ===== ===== ===== =====
    % OGRANICZENIA
    % ===== ===== ===== =====
    \subsection{Ograniczenia}

        \subsubsection{Ograniczona funkcjonalność narzędzi do sieci neuronowych w Rust}

            Mimo rosnącej popularności języka Rust\cite{Rust:popularity} i dynamicznego rozwoju jego ekosystemu, narzędzia do tworzenia modeli sztucznych sieci neuronowych są nadal w fazie rozwoju. Dostępne biblioteki w większości oferują interfejsy do istniejących rozwiązań w języku C++ lub w mniejszym stopniu są pisane od podstaw. Ze względu na wczesny etap rozwoju pierwszy typ tych narzędzi nie wykorzystuje w pełni możliwości języka Rust, a drugi typ nie zapewnia jeszcze pełnej optymalizacji procesu uczenia modeli sztucznych sieci neuronowych w porównaniu do bardziej dojrzałych rozwiązań dostępnych w językach C++ czy Python. Wykorzystana w pracy biblioteka \texttt{burn} wraz ze środowiskiem \texttt{wgpu} napotkała ograniczenia, które uniemożliwiły w pełni wykorzystanie potencjału dostępnej karty graficznej. Biblioteka na obecnym poziomie rozwoju zawiera jedynie implementacje podstawowych i klasycznych rozwiązań w dziedzinie sztucznych sieci neuronowych.

        \subsubsection{Złożoność procesu uczenia modelu sztucznej sieci neuronowej}

            Wykorzystanie dużego zbioru uczącego zawierającego milion przykładów w procesie uczenia znacząco spowolniło proces uczenia oraz strojenia modelu sztucznej sieci neuronowej. Ze względu na ograniczenia wykorzystanego narzędzia do tworzenia modelu wykonanie jednej epoki trwało około 1 godzinę na karcie graficznej NVIDIA RTX 2060 Super. Wybór tak dużego zbioru uczącego był podyktowany wysoką złożonością przestrzeni danych, której rozmiar w przypadku wykorzystania sekwencji DNA o długości 150 wynosi $4^{150}$, co daje w przybliżeniu $10^{90}$ różnych sekwencji DNA. W uczeniu kontrastowym wykorzystanie dużych zbiorów danych pozwala na stworzenie lepszego modelu, który uwzględnia więcej informacji o strukturze sekwencji.

        \subsubsection{Czasochłonność eksperymentów}

            Przeprowadzenie całego procesu klasyfikacji taksonomicznej sekwencji DNA z wykorzystaniem narzędzia \texttt{BLASTn} w ramach eksperymentów dla każdej metody i każdego podzbioru eksperymentalnego było procesem czasochłonnym. Łączny czas wykonania eksperymentów opisanych w pracy wyniósł około 5 dni, co spowodowało ograniczenie liczby eksperymentów do jednego przebiegu i rozmiaru podzbioru eksperymentalnego do maksymalnie $4096$ sekwencji.

        \subsubsection{Wykorzystanie jednego zbioru danych}

            W pracy wykorzystano jedynie część dostępnego zbioru danych, ograniczając się do jednej próbki ze względu na jej duży rozmiar. Wybór próbki zawierającej sekwencje DNA pochodzące z mikrobiomu skóry człowieka mógł ograniczyć przestrzeń analizowanych sekwencji, co mogło wpłynąć na wyniki eksperymentów, ponieważ w eksperymentach również wykorzystano sekwencje DNA pochodzące z tego samego mikrobiomu.

        \subsubsection{Złożoność obliczeniowa macierzy niepodobieństwa}

            Czas budowy macierzy niepodobieństwa rośnie wprost proporcjonalnie do kwadratu liczby sekwencji, które zostały użyte do jej budowy. Macierz niepodobieństwa wykorzystywana przez algorytm grupowania w pracy była budowana dla wszystkich sekwencji wejściowych. Zastosowane podejście znacznie ogranicza liczbę możliwych sekwencji wejściowych ze względu na szybki wzrost czasu potrzebnego na tworzenie macierzy niepodobieństwa.

    % ===== ===== ===== =====
    % MOŻLIWOŚCI DALSZEGO ROZWOJU
    % ===== ===== ===== =====
    \subsection{Możliwości dalszego rozwoju}

        \subsubsection{Zmiana architektury modelu sztucznej sieci neuronowej}

            Stworzony model sztucznej sieci neuronowej wymaga sekwencji o stałej długości, co ogranicza elastyczność modelu w analizie sekwencji o różnych długościach i wymaga tworzenia nowego modelu dostosowanego do dłuższych sekwencji w przypadku znacznej różnicy między długościami sekwencji wejściowych a oczekiwanych przez model. Dodatkowo sekwencje krótsze niż wymagane wymagają wydłużenia do zadanej długości przez wypełnienie wybraną zasadą. W przyszłości można rozważyć zmianę architektury modelu sztucznej sieci neuronowej, dostosowując ją do analizy sekwencji o zmiennych długościach poprzez zastosowanie sieci rekurencyjnych lub typu LSTM w pierwszych warstwach modelu.

        \subsubsection{Grupowanie sekwencji w paczkach}

            Ze względu na czasochłonność procesu budowy macierzy niepodobieństwa, na wstępnym etapie sekwencje wejściowe mogłyby zostać poddane losowemu grupowaniu w paczki. Następnie, w ramach każdej paczki, przeprowadzony zostałby proces wyboru reprezentantów, tak jak ma to miejsce w obecnym podejściu. Zastosowanie wstępnego grupowania umożliwiłoby ominięcie budowy jednej dużej macierzy niepodobieństwa, co pozwoliłoby na analizę większej liczby sekwencji.

        \subsubsection{Wykonywanie obliczeń równolegle}

            Możliwa jest optymalizacja niektórych operacji wykonywanych w potoku przetwarzania poprzez zastosowanie obliczeń równoległych. Pierwszym elementem, w którym można wykorzystać obliczenia równoległe, jest budowa macierzy niepodobieństwa. Macierz tą można podzielić na fragmenty, które będą budowane równolegle i niezależnie. Drugą zmianą jest modyfikacja potoku przetwarzania, która pozwalałaby na równoczesne wykonywanie niektórych etapów, bez konieczności oczekiwania na zakończenie obliczeń w innych częściach potoku. Przykładem takiego etapu jest wyliczanie zanurzeń, które mogłoby być wykorzystywane bezpośrednio do budowy macierzy niepodobieństwa bez oczekiwania na obliczenie wszystkich zanurzeń.

        \subsubsection{Automatyczny dobór liczby grup}

            Obecne podejście wymaga doboru liczby tworzonych grup przez algorytm grupowania. Liczba grup bezpośrednio determinuje liczbę reprezentantów, którzy będą poddani klasyfikacji taksonomicznej i ma wpływ na jakość oraz czas wykonania całego procesu. Możliwe byłoby zastosowanie lub stworzenie algorytmu, który automatycznie dobierałby liczbę tworzonych grup na podstawie macierzy niepodobieństwa oraz miar jakości tworzonych grup. Automatyczne dostosowanie liczby grup pozwoliłoby na zwiększenie jakości klasyfikacji w przypadku bardzo różnych sekwencji wejściowych bez spowalniania procesu dla zbiorów sekwencji bardzo podobnych.

        \subsubsection{Narzędzie do klasyfikacji taksonomicznej oparte o model sztucznej sieci neuronowej}

            Narzędzia do klasyfikacji taksonomicznej w większości opierają się na obliczaniu podobieństwa między sekwencjami wejściowymi a sekwencjami znajdującymi się w bazie danych sekwencji w celu znalezienia sekwencji, które są podobne. Wykorzystane w eksperymentach narzędzie \texttt{BLASTn} wykorzystuje $k$-mery do określenia podobieństwa między sekwencjami. 
            Grupowanie sekwencji, które zostało wykorzystane w pracy, wykonuje podobne operacje, do tych, które są wykonywane przez narzędzia do klasyfikacji taksonomicznej. Możliwe byłoby zatem stworzenie własnego narzędzia, które wykorzystywałoby bezpośrednio model sztucznej sieci neuronowej do określenia podobieństwa między sekwencjami i klasyfikacji taksonomicznej. Wykorzystanie sztucznych sieci neuronowych w przypadku wydajnej implementacji mogłoby pozwolić na osiągnięcie akceptowalnego czasu działania przy wyższej jakości.

    % ===== ===== ===== =====
    % WNIOSKI KOŃCOWE
    % ===== ===== ===== =====
    \subsection{Wnioski końcowe}

        \todo{
            \begin{enumerate}
                \item {podsumowanie głównych wniosków z pracy,}
                \item {podkręslenie najważniejszych osiągnięć}
            \end{enumerate}
        }

        \todo {
            DO NAPISANIA \\
            \textbf{Łącznie:}
            \begin{itemize}
                \item {Udało się stworzyć w pełni aplikację do przeprowadzania analiz}
                \item {Zastosowanie sztucznych sieci neuronowych poprawiło jakość klasyfikacji}
            \end{itemize}
        }