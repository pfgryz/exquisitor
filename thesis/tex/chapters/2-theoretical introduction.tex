\cleardoublepage
\section{Wstęp teoretyczny}

    % ===== ===== ===== =====
    % KLUCZOWE POJĘCIA I DEFINICJE
    % ===== ===== ===== ===== 
    \subsection{Kluczowe pojęcia i definicje}

        \subsubsection{Sekwencja DNA}

        Sekwencja DNA to sekwencja genetyczna stanowiąca zapis informacji genetycznej, który w sposób symboliczny odwzorowuje strukturę cząstki DNA, stosując alfabet złożony z czterech symboli: \texttt{A, T, C, G}. Każdy z symboli odnosi się do jednej z zasad azotowych zawartych w nukleotydach tworzących cząsteczkę DNA, odpowiednio: adeniny, tyminy, cytozyny oraz guaniny.
        Taki opis jest nazywany strukturą pierwszorzędową.

        \subsubsection{$k$-mer}

            $k$-mer to podsłowo sekwencji genetycznej o długości $k$. Dla danego alfabetu $G$ składające się z $n$ symboli istnieje dokładnie $n^k$ różnych $k$-merów o długości $k$. Sekwencja genetyczna o długości $m$ zawiera dokładnie $m - k + 1$ $k$-merów o długości $k$.


        \subsubsection{Dopasowanie sekwencji}

            Proces dopasowania sekwencji polega na wyrównaniu ich symboli w celu maksymalizacji ich wzajemnego podobieństwa, co realizowane może być poprzez wstawianie przerw (ang. \textit{gaps}) między symbolami sekwencji wejściowych. Dopasowanie sekwencji może być realizowane na dwa sposoby: lokalnie, gdzie maksymalizowane jest podobieństwo fragmentów sekwencji niezależnie od ich położenia oraz globalnie, gdzie maksymalizowane jest podobieństwo całych sekwencji. Wynikiem dopasowania są sekwencje z opcjonalnie wstawionymi przerwami oraz miara jakości dopasowania lub miara podobieństwa między sekwencjami.

        \subsubsection{Uczenie kontrastowe}
            
            Uczenie kontrastowe (ang. \textit{contrastive learning})\cite{Bromley1993} jest metodą uczenia maszynowego polegającą na maksymalizacji podobieństwa między podobnymi próbkami oraz minimalizacji podobieństwa między niepodobnymi próbkami. Wykorzystuje ona miary niepodobieństwa między reprezentacjami do nauki odróżniania kluczowych cech próbek. Zastosowanie tej metody pozwala na redukcję wymiarowości danych przy jednoczesnym zachowaniu właściwości podobieństwa i niepodobieństwa między danymi wejściowymi.

        \subsubsection{Podobieństwo kosinusowe}

            Podobieństwo kosinusowe to miara, która określa jak bardzo zbliżone, są wektory pod względem kierunku w przestrzeni wielowymiarowej. Miara wyrażona jest za pomocą kosinusa kąta między wektorami i przyjmuje wartości z zakresu $[-1; 1]$, gdzie $1$ określa pełne podobieństwo, a $-1$ całkowity brak podobieństwa. Podobieństwo kosinusowe wyrażone jest wzorem:

            \begin{equation}
                similarity_{cosine}(A, B) = \cos{\theta} = \frac{A \cdot B}{\|A\| \|B\|} = \frac{
                    \sum^{n}_{i = 1}A_i B_i
                }{
                    \sqrt{
                        \sum^{n}_{i = 1}A_i^2
                    }
                    \cdot
                    \sqrt{
                        \sum^{n}_{i = 1}B_i^2
                    }
                }
            \end{equation}

            gdzie,
            \begin{align*}
                A, B -& \text{porównywane wektory,} \\
                \theta -& \text{kąt między wektorami $A$ i $B$,} \\
                A_j, B_j -& \text{$j$-ty element wektora odpowiednio $A$ oraz $B$.}
            \end{align*}

        \subsubsection{Niepodobieństwo kosinusowe}

            Niepodobieństwo kosinusowe to miara, która określa jak bardzo wektory różnią się pod względem kierunku. Wyrażone jest wzorem:

            \begin{equation}
                dissimilarity_{cosine}(A, B) = 1 - similarity_{cosine}(A, B)
                \label{Equation:CosineDissimilarity}
            \end{equation}

            gdzie,
            \begin{align*}
                A, B -& \text{porównywane wektory,} \\
                similarity_{cosine}(A, B) -& \text{podobieństwo kosinusowe między wektorami $A$ i $B$.}
            \end{align*}

        \subsubsection{Format FASTA}

            Format FASTA jest to tekstowy format zapisu sekwencji nukleotydów. Nukleotydy reprezentowane są przez znaki \texttt{A, C, T, G}. Format składa się z jednej linii rozpoczynającej się znakiem \texttt{>} zawierającej identyfikator i opcjonalny opis oraz kolejnych linii zawierających sekwencję nukleotydów. Możliwy jest zapis wielu sekwencji w formacie FASTA w jednym pliku poprzez połączenie wielu sekwencji zapisanych w tym formacie.

        \subsubsection{Format FASTQ} 

            Format FASTQ jest to tekstowy format zapisu sekwencji nukleotydów, który zawiera dodatkowo informacje o jakości sekwencjonowania. Format umożliwia zapis wielu sekwencji w jednym pliku. Zapis każdej sekwencji składa się z czterech części: 
            \begin{enumerate}
                \item {
                    Pierwsza część to linia rozpoczynająca się od znaku \texttt{@} i zawierająca identyfikator oraz opcjonalny opis sekwencji.
                }
                \item {
                    Druga część to linie zawierające sekwencję nukleotydów, reprezentowanych przez znaki \texttt{A, C, T, G}.
                }
                \item {
                    Trzecia część to znak \texttt{+}, który oddziela sekwencję od informacji o jakości.
                }
                \item {
                    Czwarta część to linie zawierające zapis informacji o jakości sekwencjonowania. Informacja ta jest zakodowana za pomocą pojedynczych znaków ASCII, które odpowiadają jakości poszczególnych nukleotydów w sekwencji.
                }
            \end{enumerate}

        \subsubsection{\texttt{BLASTn}}

            \texttt{BLASTn} (ang. \textit{Basic Local Alignment Search Tool for Nucleotides}) to wariant narzędzia bioinformatycznego \texttt{BLAST}, zaprojektowany do porównywania sekwencji nukleotydów. Umożliwia wyszukiwanie podobieństw między analizowanymi sekwencjami nukleotydów a sekwencjami znajdującymi się w wybranej bazie sekwencji nukleotydów. Może być wykorzystywany do przeprowadzania klasyfikacji taksonomicznej.

        \subsubsection{Klasyfikacja taksonomiczna}

            Klasyfikacja taksonomiczna sekwencji DNA to proces przypisywania sekwencji DNA do określonych grup taksonomicznych takich jak gatunki czy rodzaje, na podstawie ich podobieńśtwa do znanych sekwencji w bazach sekwencji genetycznych.

        \subsubsection{Algorytm k-medoidów}

            Algorytm k-medoidów to metoda grupowania danych, w której dane podzielone są na $k$ grup, a reprezentantem każdej grupy jest rzeczywisty punkt danych, zwany medoidem. Algorytm znajduje zastosowanie w problemach, w których nie można wyznaczyć średniej lub centroidu np. w przypadku danych kategorycznych.

    % ===== ===== ===== =====
    % PRZEGLĄD LITERATURY
    % ===== ===== ===== ===== 
            \subsection{Przegląd literatury}

            \todo{RN: przeniosłem z początku rozdziału, bo tutaj lepiej pasuje, do złączenia z kolejnym podrozdziałem}

        \todo{
            \begin{enumerate}
                \item {podsumowanie dotychczasowych badań, publikacji i prac naukowych,}
                \item {krótkie przedstawienie osiągnięć w analizowanych obszarze,}
                \item {wskazanie luki badawczej}
            \end{enumerate}
        }


    % ===== ===== ===== =====
    % METODY I PODEJŚCIA
    % ===== ===== ===== ===== 
    \subsection{Metody i podejścia}

        \todo{
            \begin{enumerate}
                \item {Przedstawienie najpopularniejszych metod, podejść lub algorytmów wykorzystywanych w literaturze przedmiotu w kontekście tematu pracy.,}
                \item {Opis sposobów, w jaki te metody zostały zaadoptowane lub zmodyfikowane w pracy.}
                \item {Wskazanie zalet i wad poszczególnych metod.}
            \end{enumerate}
        }

        \todo {
            Kraken2 itd.
        }

        \todo {
            Needleman-Wunsch
        
            Wady: wolne określanie podobieństwa 
            Zalety: uwzględnia strukturę
        }   

        \todo {
            KMer

            Wady: nie uwzględnia struktury
            Zalety: szybkie, łatwe w wykorzystaniu, SOTA
        }

    % ===== ===== ===== =====
    % TEORETYCZNE PODSTAWY PROBLEMU
    % ===== ===== ===== ===== 
    \subsection{Teoretyczne podstawy problemu}

        \todo{
            \begin{enumerate}
                \item {Szczegółowe przedstawienie teoretycznych podstaw, które stanowią tło dla rozwiązania danego problemu w pracy,}
                \item {Omówienie teorii, które są bezpośrednio związane z problemem badawczym (np. modele matematyczne, algorytmy, teorie z dziedziny nauk komputerowych, inżynierii, matematyki).}
            \end{enumerate}
        }
            
%         \subsubsection{Algorytm Needlema-Wunscha}
        
%            Algorytm Needlemana-Wunscha stanowi metodę wykorzystywaną do ustalania globalnego dopasowania pomiędzy dwiema sekwencjami \cite{NeedlemanWunsch1970}. Metoda ta polega na skonstruowaniu macierzy podobieństwa pomiędzy sekwencjami zgodnie z ustalonymi regułami:

%            \begin{equation}
%                 \begin{aligned}
%                     D_{i,0} &= i \cdot g, & \text{dla } & i \in [0, n] \\
%                     D_{0,j} &= j \cdot g, & \text{dla } & j \in [1, m] \\
%                     D_{i,j} &= \max
%                     \begin{cases}
%                     D_{i - 1, j} + g \\
%                     D_{i, j - 1} + g \\
%                     D_{i - 1, j - 1} + s(A_i, B_j)
%                     \end{cases}, & \text{dla } & i \in (0, n] \text{ oraz } j \in (0, m]
%                 \end{aligned}
%                 \label{Equation:NeedlemanWunsch}
%             \end{equation}

%             gdzie,
%             \begin{align*} 
%                 & g, s(A_i, B_j) \in \mathbb{R} \\
%                 A, B -& \text{porównywane sekwencje}, \\
%                 n, m -& \text{długości sekwencji } A \text{ oraz } B, \\
%                 D -& \text{macierz podobieństwa o rozmiarach } n \text{ x } m, \\
%                 g -& \text{kara za przerwę}, \\
%                 s(A_i, B_j) -& \text{podobieństwo między  } i\text{-tym elementem w sekwencji A,} \\ 
%                 & \text{a } j \text{-tym elementem w sekwencji B}. \\
%             \end{align*}

%             Wartość znajdująca się w $D_{n, m}$ określa liczbowo jakość globalnego dopasowania sekwencji.
