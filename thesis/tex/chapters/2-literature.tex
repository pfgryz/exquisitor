\clearpage
\section{Analiza literatury}

**TODO** [ok. 15 stron]

\begin{enumerate}
    \item Opis problemu
    \item Opis rozwiązań problemu
    \item Opis algorytmów rozwiązujących problem
    \item Opis narzędzi (tutaj: Rust, Python itd.) + używane biblioteki
\end{enumerate}

Wprowadzenie do zagadnienia
\begin{itemize}
    \item Duża liczba organizmów
    \item Nowe metody "masowego" sekwencjowania
\end{itemize}

Opis problemu, przeglądówki
\begin{itemize}
    \item https://arxiv.org/pdf/1510.06621 - Przeglądówka różnych narzędzi stosowanych do np. klasyfikacji czy klastrowania
    \item https://arxiv.org/pdf/1808.01038 - Wyszukiwanie sekwencji - przeglądówka po historii
    \item https://arxiv.org/html/2408.12751v1 - Storage clustering
    \item https://arxiv.org/pdf/1006.4114 - DNA search engine
\end{itemize}

Rozwiazania:
\begin{itemize}
    \item https://arxiv.org/abs/1306.1569 - znajdowanie kilku representatów na podstawie losowo wybranych odczytów. Porównanie pozostałych sekwencji z wybranymi reprezentatami
    \item https://arxiv.org/abs/2407.02538 - zamiana DNA w mapę 2D, wykonanie algorytmów DL na mapach 2D w celu obliczania odległości bez alignmentu oraz klastrowania
    \item https://arxiv.org/abs/2111.09656 - Contrastive learning binning
\end{itemize}

Opis algorytmów
\begin{itemize}
    \item k-mer clustering
    \item jeden z algorytmów do alignmentu
    \item Contrastive learning
    \item Indeks Jaccarda
    \item Baza danych Blast
\end{itemize}