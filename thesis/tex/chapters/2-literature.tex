\clearpage
\section{Analiza literatury}

KONIECZNIE DODAĆ:
https://cami-challenge.org/
https://cami-challenge.org/datasets/Toy%20Human%20Microbiome%20Project/
cami-challenge.org/datasets/Toy%20Human%20Microbiome%20Project/
- sample 1 - skin sample

**TODO** [ok. 15 stron]

\begin{enumerate}
    \item Opis problemu
    \item Opis rozwiązań problemu
    \item Opis algorytmów rozwiązujących problem
    \item Opis narzędzi (tutaj: Rust, Python itd.) + używane biblioteki
\end{enumerate}

Wprowadzenie do zagadnienia
\begin{itemize}
    \item Duża liczba organizmów
    \item Nowe metody "masowego" sekwencjowania
\end{itemize}

Opis problemu, przeglądówki
\begin{itemize}
    \item https://arxiv.org/pdf/1510.06621 - Przeglądówka różnych narzędzi stosowanych do np. klasyfikacji czy klastrowania
    \item https://arxiv.org/pdf/1808.01038 - Wyszukiwanie sekwencji - przeglądówka po historii
    \item https://arxiv.org/html/2408.12751v1 - Storage clustering
    \item https://arxiv.org/pdf/1006.4114 - DNA search engine
\end{itemize}

Rozwiazania:
\begin{itemize}
    \item https://arxiv.org/abs/1306.1569 - znajdowanie kilku representatów na podstawie losowo wybranych odczytów. Porównanie pozostałych sekwencji z wybranymi reprezentatami
    \item https://arxiv.org/abs/2407.02538 - zamiana DNA w mapę 2D, wykonanie algorytmów DL na mapach 2D w celu obliczania odległości bez alignmentu oraz klastrowania
    \item https://arxiv.org/abs/2111.09656 - Contrastive learning binning
\end{itemize}

Opis algorytmów
\begin{itemize}
    \item k-mer clustering      -> k-medoids clustering
    \item jeden z algorytmów do alignmentu
    \item Contrastive learning
    \item Indeks Jaccarda
    \item Baza danych Blast
    \item Needleman-Wunsch Algorithm - https://pubmed.ncbi.nlm.nih.gov/5420325/
\end{itemize}


Opis narzędzi

# Python
- {{ krótki opis pythona }}

## Biblioteki
- psutil - do monitorowania zużycia procesora oraz pamięci przez eksperymenty
- typing (wbudowana) - {{ opis }}
- subprocess (wbudowana) - do uruchamiania eksperymentów w nowych procesach - w celu możliwości monitorowania zużycie przez nie zasobów

# Rust

## Biblioteki

Frontend
- askama - (Jinja template engine) - wykorzystywana do eegenerowania zawartości (SSR - Server side rendering) stron internetowych
- tower -
- sqlx - wykorzystywana do łączenia się z bazą danych
- dotenv - wczytywanie zmiennych środowiskowych z pliku
- tracing - śledzenie i instrumentowanie aplikacji - wyświetlanie informacji na temat stanu aplikacji (np. requestów czy błędów)
- tokio - biblioteka implementująca operacje asynchroniczne na plikach oraz pozwalająca na współbieżność realizacji zadań
- axum - framework tworzenia aplikacji webowych

Backend
- serde - wykorzystywane do serializacji oraz deserializacji danych w celu zapisania do pliku
- serde_json - serializacja do JSON
- tempfile - tworzenie plików tymczasowych, wykorzystywane do testowania aplikacji
- float_cmp - wykorzystywane do testów, pozwala na porównywanie liczb zmiennoprzecinkowych
- num_traits - wykorzystanie w celu opisania generycznych funkcji do obliczania np. dystansu euklidesowego

# HTML, CSS, JS
- realizacja strony internetowej
- (brak bibliotek zewnętrznych)




% ----------------------------
%     1   OPIS PROBLEMU
% ----------------------------




Historia

    Needleman and Wunsch (1970):


% ----------------------------
%     2   OBECNE ROZWIĄZANIA
% ----------------------------

- Klastrowanie sekwencji z wykorzystaniem alignmentu
    https://pmc.ncbi.nlm.nih.gov/articles/PMC338945/
- Klastrowanie sekwencji z wykorzystaniem k-merów


- CD-HIT
- Mothur


% ----------------------------
%     3   OPIS ALGORYTMÓW
% ----------------------------

1. Algorytm Needleman-Wunscha
2. Odległość k-mer



- definicja problemu


RZECZYWISTY PRZEGLĄD LITERATURY

--------------------------------------
- powiększanie się baz danych sekwencji
- zwiększenie się ilości i przepustowośc sekwencjowania sekwencji


- metagenomika - wprowadzenie
    - intensowne wyszukiwanie w bazach danych
    - wzrost czasochłonności wyszukiwania

- definicja problemu

-----------------------------------------

- metody
- wady i zalety tych metody
- poszukiwania rozwiązania

------------------------------------------

Opisać moje rozwiązanie i pokazac różnice


WSTEP TEORETYCZNY
    - wprowadzenie podstawowych pojęc
    - wprowadzenie algorytmów
    - narzędzia np. Python, Rust
        - biblioteki
    - wykorzystane algorytmy w potoku - (wspomnieć i z grubsza opisać działanie)











