\cleardoublepage{}

\secondabstract{}
The aim of this thesis is to develop an application for analyzing genetic material DNA from various organisms collected from the environment. The application uses a database of genetic sequences and introduced an improvement in the analysis process by using various methods, including machine learning.
Taxonomic classification of genetic sequences was used to perform genetic material analysis. Optimization of the process was achieved by clustering genetic sequences and selecting representatives of the sequences that were used for taxonomic classification. The thesis implemented two classical methods: one using sequence alignments and the other based on $k$-mers. An original method was also developed, based on machine learning, using artificial neural networks with contrastive learning. A console application was created to conduct the analysis process, and a web application was developed to allow users to submit genetic material analysis requests. The implementation was done in the Rust programming language.
Qualitative experiments were conducted, comparing the implemented methods against a full taxonomic classification of all sequences, as well as performance experiments, evaluating the execution time of the entire process. The results showed that the custom method demonstrated the highest classification quality and the fastest execution time compared to the other methods.

Abstract

\secondkeywords{}
DNA sequences, taxonomic classification,  metagenomics, artificial neural networks, contrastive learning
