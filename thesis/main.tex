%-----------------------------------------------
%  Engineer's & Master's Thesis Template
%  Copyleft by Artur M. Brodzki & Piotr Woźniak
%  Warsaw University of Technology, 2019-2022
%-----------------------------------------------

\documentclass[
    bindingoffset=5mm,  % Binding offset
    footnoteindent=3mm, % Footnote indent
    hyphenation=true    % Hyphenation turn on/off
]{src/wut-thesis}

% ---- ---- ---- ----
% Packages
% ---- ---- ---- ----
\usepackage{tikz}
\usepackage{float}
\usepackage{longtable}
\usepackage{tabularx}
\usepackage{xltabular}
\usepackage{booktabs}
\usepackage{tabularray}

\usepackage{todonotes}
\setuptodonotes{inline}


% ---- ---- ---- ----
% Packages Configuration
% ---- ---- ---- ----
\usetikzlibrary{shapes.geometric, arrows.meta, positioning}
\DefTblrTemplate{caption-tag}{default}{\small \textbf{Tabela\hspace{0.25em}\thetable}}
\DefTblrTemplate{caption-sep}{default}{.\enskip}
\DefTblrTemplate{caption-text}{default}{\small \InsertTblrText{caption}}
\DefTblrTemplate{contfoot-text}{default}{Kontynuacja na następnej stronie.}
\DefTblrTemplate{conthead-text }{default}{}

\newcolumntype{R}{>{\raggedleft\arraybackslash}X}

% % ---- ---- ---- ----
% % Writing utilities
% % ---- ---- ---- ----
% \newcommand{\todo}[1]{%
%     \definecolor{Red}{HTML}{ff6b6b}
%     \tikzstyle{box} = [rectangle, minimum width=15cm, minimum height=1cm, line width=2pt, text centered, draw=Red, align=left, inner sep=10pt, Red]

%     \begin{tikzpicture}
%         \node [box] {
%             \textbf{Ta sekcja wymaga rozwinięcia}\\%
%             \parbox{\dimexpr\linewidth-2em}{#1}%
%         };
%     \end{tikzpicture}
% }
\newcommand{\question}[1]{%
    \definecolor{Blue}{HTML}{54a0ff}
    \tikzstyle{box} = [rectangle, minimum width=15cm, minimum height=1cm, line width=2pt, text centered, draw=Red, align=left, inner sep=10pt, Blue]

    \begin{tikzpicture}
        \node [box] {
            \textbf{Pytanie}\\%
            \parbox{\dimexpr\linewidth-2em}{#1}%
        };
    \end{tikzpicture}
}

% ---- ---- ---- ----
% Title page
% ---- ---- ---- ----
\graphicspath{{tex/imagesg/}}
\addbibresource{bibliography.bib}

% ---- ---- ---- ----
% Thesis template configuration
% ---- ---- ---- ----
\facultyeiti{}
\EngineerThesis{}
\langpol{}

\begin{document}

% ---- ---- ---- ----
% Title page
% ---- ---- ---- ----
\instytut{Informatyki}
\kierunek{Informatyka}
\specjalnosc{Sztuczna Inteligencja}
\title{
    Aplikacja do monitorowania DNA w środowisku, wykorzystująca nowy algorytm wyboru reprezentantów za pomocą sztucznych sieci neuronowych
}

% English title
\engtitle{
    Application for DNA monitoring in the environment using a novel representative selection algorithm based on artificial neural networks
}

% Polish title
\poltitle{
    Aplikacja do monitorowania DNA w środowisku, wykorzystująca nowy algorytm wyboru reprezentantów za pomocą sztucznych sieci neuronowych
}
\author{Patryk Filip Gryz}
\album{318657}
\promotor{dr hab.\ inż. Robert Nowak}
\date{\the\year}
\maketitle

% ---- ---- ---- ----
% Abstract in polish
% ---- ---- ---- ----
\cleardoublepage{}

\abstract{}
Celem niniejszej pracy dyplomowej jest stworzenie aplikacji służącej do przeprowadzania analiz materiału genetycznego DNA pochodzącego od różnych organizmów, pobranego ze środowiska. Aplikacja wykorzystuje bazę danych sekwencji genetycznych oraz wprowadza usprawnienie w procesie analizy dzięki zastosowaniu różnych metod, w tym uczenia maszynowego.
Wykorzystano klasyfikację taksonomiczną sekwencji genetycznych do przeprowadzania analizy materiału genetycznego. Optymalizacja procesu została osiągnięta poprzez grupowanie sekwencji genetycznych oraz wybór reprezentantów sekwencji, które posłużyły do klasyfikacji taksonomicznej. W ramach pracy zaimplementowano dwie metody klasyczne: wykorzystującą wyrównania sekwencji oraz analizę opartą na $k$-merach. Opracowano również autorską metodę opartą na uczeniu maszynowym wykorzystującą sztuczne sieci neuronowe wraz z uczeniem kontrastowym. Stworzono aplikację konsolową, przeprowadzającą proces analizy, oraz aplikację internetową umożliwiającą użytkownikowi zlecanie analiz materiału genetycznego. Implementację stworzono w języku Rust.
Przeprowadzono eksperymenty jakościowe, w których porównano zaimplementowane metody względem pełnej klasyfikacji taksonomicznej wszystkich sekwencji, oraz eksperymenty wydajnościowe, oceniające czas wykonania całego procesu. Wyniki wykazały, że opracowana metoda autorska cechowała się najwyższą jakością klasyfikacji oraz najszybszym czasem działania w porównaniu z pozostałymi metodami.

\keywords{}
sekwencje DNA, klasyfikacja taksonomiczna, metagenomika, sztuczne sieci neuronowe, uczenie kontrastowe



% ---- ---- ---- ----
% Abstract in english
% ---- ---- ---- ----
\cleardoublepage{}

\secondabstract{}

Abstract
\todo{TODO}

\secondkeywords{}
DNA sequences, taxonomic classification,  metagenomics, artificial neural networks, contrastive learning


\pagestyle{plain}

\makeauthorship

% ---- ---- ---- ----
% Table of contents
% ---- ---- ---- ----
\cleardoublepage{}

\setcounter{tocdepth}{2}

\tableofcontents

% ---- ---- ---- ----
% Chapters
% ---- ---- ---- ----
\cleardoublepage{}
\pagestyle{headings}

\clearpage
\section{Wprowadzenie}

**TODO** [ok. 3 strony]

\begin{enumerate}
    \item Krótki opis problemu [akapit]
    \item Cel pracy (co zamierzam zrobić) [1 akapit]
    \item Zakres pracy (jak zamierzam zrobić) [1 akapit]
    \item Układ pracy (co w którym rozdziale) [1 akapit]
\end{enumerate}

\clearpage
\section{Wstęp teoretyczny}

    % ===== ===== ===== =====
    % PRZEGLĄD LITERATURY
    % ===== ===== ===== ===== 
    \subsection{Przegląd literatury}

        \todo{
            \begin{enumerate}
                \item {podsumowanie dotychczasowych badań, publikacji i prac naukowych,}
                \item {krótkie przedstawienie osiągnięć w analizowanych obszarze,}
                \item {wskazanie luki badawczej}
            \end{enumerate}
        }

    % ===== ===== ===== =====
    % KLUCZOWE POJĘCIA I DEFINICJE
    % ===== ===== ===== ===== 
    \subsection{Kluczowe pojęcia i definicje}

        \todo{
            \begin{enumerate}
                \item {wyjaśnienie podstawowych pojęć,}
                \item {omówienie istotnych teorii, które stanowią fundament rozwiązań lub badań w pracy.,}
            \end{enumerate}
        }

    % ===== ===== ===== =====
    % METODY I PODEJŚCIA
    % ===== ===== ===== ===== 
    \subsection{Metody i podejścia}

        \todo{
            \begin{enumerate}
                \item {Przedstawienie najpopularniejszych metod, podejść lub algorytmów wykorzystywanych w literaturze przedmiotu w kontekście tematu pracy.,}
                \item {Opis sposobów, w jaki te metody zostały zaadoptowane lub zmodyfikowane w pracy.}
                \item {Wskazanie zalet i wad poszczególnych metod.}
            \end{enumerate}
        }

    % ===== ===== ===== =====
    % TEORETYCZNE PODSTAWY PROBLEMU
    % ===== ===== ===== ===== 
    \subsection{Teoretyczne podstawy problemu}

        \todo{
            \begin{enumerate}
                \item {Szczegółowe przedstawienie teoretycznych podstaw, które stanowią tło dla rozwiązania danego problemu w pracy,}
                \item {Omówienie teorii, które są bezpośrednio związane z problemem badawczym (np. modele matematyczne, algorytmy, teorie z dziedziny nauk komputerowych, inżynierii, matematyki).}
            \end{enumerate}
        }


% %region TIKZ Configuration



% %endregion

% \clearpage
% \section{Wstęp teoretyczny}

%     \subsection{Podstawowe definicje}

%         \subsubsection{Sekwencja DNA}

%             Sekwencja DNA stanowi zapis genetyczny, który w sposób symboliczny odwzorowuje strukturę cząsteczki DNA, stosując alfabet złożony z czterech symboli: $A, T, C, G$. Każdy z symboli odnosi się do jednej z zasad azotowych zawartych w nukleotydach tworzących cząsteczkę DNA odpowiednio: adeniny, tyminy, cytozyny oraz guaniny.

%             \subsection{KMer}
        
%             $K$—mery są to podsłowa sekwencji genetycznej o długości $k$. Dla danego alfabetu $G$ składającego się z $n$ symboli istnieje $n^k$ różnych $k$-merów o długości $k$. Sekwencja genetyczna o długości $n$ zawiera dokładnie $n - k + 1$ $k$-merów o długości $k$.

%         \subsubsection{Dopasowanie sekwencji}
        
%             Proces dopasowywania sekwencji polega na wyrównywaniu ich symboli w celu maksymalizacji ich wzajemnego podobieństwa, co realizowane jest poprzez wstawianie przerw (ang. \textit{gaps}).
            
%         \subsubsection{Algorytm Needlema-Wunscha}
        
%            Algorytm Needlemana-Wunscha stanowi metodę wykorzystywaną do ustalania globalnego dopasowania pomiędzy dwiema sekwencjami \cite{NeedlemanWunsch1970}. Metoda ta polega na skonstruowaniu macierzy podobieństwa pomiędzy sekwencjami zgodnie z ustalonymi regułami:

%            \begin{equation}
%                 \begin{aligned}
%                     D_{i,0} &= i \cdot g, & \text{dla } & i \in [0, n] \\
%                     D_{0,j} &= j \cdot g, & \text{dla } & j \in [1, m] \\
%                     D_{i,j} &= \max
%                     \begin{cases}
%                     D_{i - 1, j} + g \\
%                     D_{i, j - 1} + g \\
%                     D_{i - 1, j - 1} + s(A_i, B_j)
%                     \end{cases}, & \text{dla } & i \in (0, n] \text{ oraz } j \in (0, m]
%                 \end{aligned}
%                 \label{Equation:NeedlemanWunsch}
%             \end{equation}

%             gdzie,
%             \begin{align*} 
%                 & g, s(A_i, B_j) \in \mathbb{R} \\
%                 A, B -& \text{porównywane sekwencje}, \\
%                 n, m -& \text{długości sekwencji } A \text{ oraz } B, \\
%                 D -& \text{macierz podobieństwa o rozmiarach } n \text{ x } m, \\
%                 g -& \text{kara za przerwę}, \\
%                 s(A_i, B_j) -& \text{podobieństwo między  } i\text{-tym elementem w sekwencji A,} \\ 
%                 & \text{a } j \text{-tym elementem w sekwencji B}. \\
%             \end{align*}

%             Wartość znajdująca się w $D_{n, m}$ określa liczbowo jakość globalnego dopasowania sekwencji.

%         \subsubsection{Uczenie kontrastowe}

%             Uczenie kontrastowe (ang. \textit{contrastive learning}) \cite{Bromley1993} jest metodą polegającą na nauce reprezentacji danych poprzez porównywanie i różnicowanie podobnych oraz różnych przykładów. Dzięki zastosowaniu tej techniki, reprezentacje danych zachowują właściwości podobieństwa i różnicy między danymi, które reprezentują.

%         \subsubsection{Podobieństwo i niepodobieństwo kosinusowe}

%             Podobieństwo i niepodobieństwo kosinusowe są miarami, które mogą być wykorzystywane do porównywania wektorów liczbowych, zdefiniowane one są wzorami:

%             \begin{equation}
%                 similarity_{cosine} = \cos{\theta} = \frac{A \cdot B}{\|A\| \|B\|} = \frac{
%                         \sum^{n}_{i = 1}A_i B_i
%                     }{
%                         \sqrt{
%                             \sum^{n}_{i = 1}A_i^2
%                         }
%                         \cdot
%                         \sqrt{
%                             \sum^{n}_{i = 1}B_i^2
%                         }
%                     }
%             \end{equation}

%             \begin{equation}
%                 dissimilarity_{cosine}(A, B) = 1 - similarity_{cosine}(A, B)
%             \end{equation}

%             gdzie,
%             \begin{align*} 
%                 A, B -& \text{porównywane wektory}, \\
%                 \theta -& \text{kąt między wektorami $A$ i $B$}, \\
%                 A_j, B_j -& \text{$j$-ty element wektora odpowiednio $A$ oraz $B$}.
%             \end{align*}


%     \subsection{Metody}
    
%             W pracy zaimplementowano trzy metody: nową metodę opartą na sieciach neuronowych oraz dwie klasyczne metody wykorzystywane do porównania. Nowa metoda oparta na sieciach neuronowych została opracowana w celu połączenia zalet obu klasycznych metod oraz eliminacji ich niedoskonałości. W szczególności skoncentrowano się na przyspieszeniu procesu określania niepodobieństwa w porównaniu do algorytmu Needlemana-Wunscha oraz na zwiększeniu jakości mierzonych niepodobieństw między sekwencjami DNA w porównaniu z metodą wykorzystującą $k$-mery, która nie bierze pod uwagę w pełni struktury porównywanych sekwencji.
    
%             % Needleman-Wunsch
%             \subsubsection{Zmodyfikowany algorytm Needlemana-Wunscha}
            
%                 Pierwszą metodą klasyczną, która została wykorzystana do określania niepodobieństwa między sekwencjami DNA jest zmodyfikowany algorytm Needlama-Wunscha. Algorytm pozwala na dokładne określanie niepodobieństwa z uwzględnieniem przerw oraz zmiany danych zasad. Modyfikacja algorytmu polega na zmianie budowy macierzy podobieństwa oraz wprowadzeniu dodatkowych ograniczeń w celu zapewnienia, że niepodobieństwo będzie mieściło się w przedziale: $[0, \infty)$:
    
%                 \begin{equation}
%                     \begin{aligned}
%                         D_{i,j} &= \min
%                         \begin{cases}
%                         D_{i - 1, j} + g \\
%                         D_{i, j - 1} + g \\
%                         D_{i - 1, j - 1} + s(A_i, B_j)
%                         \end{cases}, & \text{dla } & i \in (0, n] \text{ oraz } j \in (0, m] \\
%                         & g, s(A_i, B_j) \in \mathbb{R}^{+}
%                     \end{aligned}
%                 \end{equation}
    
%                 Wartość znajdująca się w $D_{n, m}$ jest miarą niepodobieństwa między sekwencjami.
                
%             % KMer
%             \subsubsection{KMer}
            
%                 Druga zaimplementowana metoda wykorzystuje $k$-mery do określania niepodobieństwa między sekwencjami DNA, poprzez wyznaczenie odległości euklidesowej w przestrzeni $k$-merów, gdzie osie tej przestrzeni odpowiadają różnym $k$-merom występującym w sekwencjach, a wektory reprezentują ich częstości wystąpień. Niepodobieństwo wyliczone jest według wzoru:
    
%                 \begin{equation}
%                     dissimilarity_{kmer}(A, B, k) = \sqrt{\sum_{m \in M_{k}} (A_m - B_m)^{2}}
%                 \end{equation}
    
%                 gdzie,
%                 \begin{align*} 
%                     k -& \text{długość $k$-merów}, \\
%                     A, B -& \text{porównywane sekwencje}, \\
%                     M -& \text{zbiór wszystkich możliwych sekwencji genetycznych o długości $k$}, \\
%                     A_j, B_j -& \text{liczba wystąpień sekwencji } j \text{ odpowiednio w sekwencjach } A \text{ i } B. \\
%                 \end{align*}
            
%             % Sztuczna sieć neuronowa
%             \subsubsection{Sztuczna sieć neuronowa}
            
%                 Zaproponowana metoda wykorzystuje sztuczne sieci neuronowe wraz z uczeniem kontrastowym do stworzenia modelu, który będzie pozwalał na tworzenie reprezentacji sekwencji wejściowych, które zachowają właściwości niepodobieństwa między sekwencjami. Na podstawie stworzonych reprezentacji sekwencji będzie wyliczone niepodobieństwo z wykorzystaniem niepodobieństwa kosinusowego.
    
%                 \subsubsection{Architektura}
    
%                     Architektura sieci neuronowej składa się z 2 bloków wykorzystujących warstwy splotowe, które odpowiadają za ekstrakcję niskopoziomowych cech sekwencji, warstwy spłaszczającej oraz trzech warstw perceptronów wielowarstwowych, które odpowiadają za stworzenie reprezentacji na podstawie wyekstrahowanych cech. Wyjściem jest wektor zanurzeń o rozmiarze $64$. Schematycznie architektura została przedstawiona na rysunku ~\ref{Architektura}).
    
%                     \begin{figure}[H]
%                         \begin{center}
%                             {
% ===== BEGIN =====
% ----- -----
% COLORS
% ----- -----
\definecolor{Green}{HTML}{1dd1a1}   % Input
\definecolor{Blue}{HTML}{54a0ff}    % Linear
\definecolor{Yellow}{HTML}{feca57}  % Convolution
\definecolor{Purple}{HTML}{5f27cd}  % Batch Norm
\definecolor{Grey}{HTML}{576574}    % Dropout
\definecolor{Red}{HTML}{ff6b6b}     % Output
\definecolor{Pink}{HTML}{ff9ff3}    % Activation
\definecolor{Background}{HTML}{c8d6e5}

% ----- -----
% ELEMENTS
% ----- -----
\tikzstyle{box} = [rectangle, rounded corners, minimum width=5cm, minimum height=1cm, text centered, draw=black, align=center]
\tikzstyle{input} = [box, fill=Green!30]
\tikzstyle{linear} = [box, fill=Blue!30]
\tikzstyle{conv} = [box, fill=Yellow!30]
\tikzstyle{bn} = [box, fill=Purple!30]
\tikzstyle{activation} = [box, fill=Pink!30]
\tikzstyle{dropout} = [box, fill=Grey!30]
\tikzstyle{output} = [box, fill=Red!30]

\tikzstyle{arrow} = [very thick, -Triangle]
\tikzstyle{arrow:text} = [pos=0.5, right, font=\footnotesize]

% ----- -----
% PICTURE
% ----- -----
\begin{tikzpicture}[node distance=2cm]
    \node (input) [input] { Wejście };
    \node (conv1) [conv, below of=input] { Splot 1D \\ \textbf{16@1x16, krok: 4} };
    \node (bn1) [bn, below of=conv1] { Normalizacja wsadowa };
    \node (conv2) [conv, below of=bn1] { Splot 1D \\ \textbf{32@1x8} };
    \node (bn2) [bn, below of=conv2] { Normalizacja wsadowa };
    \node (flatten) [linear, below of=bn2] { Warstwa spłaszczająca };
    \node (flatten-right) [right of=flatten, xshift=2cm] {};
    
    \node (fc1-left) [right of=input, xshift=2cm] {};
    \node (fc1) [linear, right of=input, xshift=6cm] { Warstwa gęsta };
    \node (act1) [activation, below of=fc1] { Aktywacja \\ \textbf{GELU} };
    \node (drop1) [dropout, below of=act1] { Wyłączenie neuronów };
    \node (fc2) [linear, below of=drop1] { Warstwa gęsta };
    \node (act2) [activation, below of=fc2] { Aktywacja \\ \textbf{GELU} };
    \node (drop2) [dropout, below of=act2] { Wyłączenie neuronów };
    \node (fc3) [linear, below of=drop2] { Warstwa gęsta };
    \node (output) [output, below of=fc3] { Wyjście };

    \draw [arrow] (input) -- (conv1) node [arrow:text] {1x600};
    \draw [arrow] (conv1) -- (bn1) node [arrow:text] {16x147};
    \draw [arrow] (bn1) -- (conv2) node [arrow:text] {16x147};
    \draw [arrow] (conv2) -- (bn2) node [arrow:text] {32x140};
    \draw [arrow] (bn2) -- (flatten) node [arrow:text] {32x140};

    \draw [arrow] (flatten.east) -- (flatten-right) -- node [arrow:text] {1x4480} (fc1-left) -- (fc1.west);
    \draw [arrow] (fc1) -- (act1) node [arrow:text] {1x4096};
    \draw [arrow] (act1) -- (drop1) node [arrow:text] {1x4096};
    \draw [arrow] (drop1) -- (fc2) node [arrow:text] {1x4096};
    \draw [arrow] (fc2) -- (act2) node [arrow:text] {1x512};
    \draw [arrow] (act2) -- (drop2) node [arrow:text] {1x512};
    \draw [arrow] (drop2) -- (fc3) node [arrow:text] {1x512};
    \draw [arrow] (fc3) -- (output) node [arrow:text] {1x64};
\end{tikzpicture}

% ===== END =====
}
%                         \end{center}
%                         \caption{
%                             Schemat architektury sieci neuronowej.
%                         } 
%                         \label{Architektura}
%                     \end{figure}
    
%                 \subsubsection{Przykłady uczące}
%                     Przykłady uczące składają się z kotwicy (ang. \textit{anchor}) oraz dwóch sekwencji: pierwsza sekwencja podobna do kotwicy, druga nie.
    
%                 \subsubsection{Zbiór danych}
%                     Do stworzenia zbioru danych: treningowego oraz walidacyjnego została wykorzystana pierwsza próbka sekwencji genetycznych ze zbioru \textit{CAMI II Toy Human Microbiome Project} \cite{Fritz2019}. Zbiory zostały uzyskane poprzez losowy wybór sekwencji genetycznych z próbki, które następnie zostały uznane za kotwice. Przykłady podobne oraz niepodobne uzyskano poprzez modyfikację kotwicy odpowiednio w zakresie $[0; 0.2]$, $[0.2; 0.8]$.
    
%                 \subsubsection{Proces uczenia}
    
%                     Przeprowadzono proces uczenia na zbiorze $10^{6}$ przykładów treningowych oraz $10^{4}$ walidacyjnych.
    
%                     \paragraph{Funkcja straty}
    
%                         Wykorzystano funkcję straty zdefiniowaną jako:
                        
%                         \begin{equation}
%                             \text{Strata kontrastowa} = [m_{pos} - s_{pos}]_{+} + [s_{neg} - m_{neg}]_{+}
%                         \end{equation}
    
%                         gdzie,
%                         \begin{align*}
%                             m_{pos}, m_{neg} &- \text{margines podobieństwa między przykładami pozytywnymi a kotwicą,} \\
%                             &\text{oraz między przykładami negatywnymi a kotwicą}, \\
%                             s_{pos}, s_{neg} &- \text{podobieństwo kosinusowe przykładu pozytywnego do kotwicy,} \\
%                             &\text{oraz negatywnego do kotwicy.}
%                         \end{align*}
    
%                     \paragraph{Optymalizator}
                    
%                         W procesie uczenia wykorzystano optymalizator \textit{AdamW} \cite{Loshchilov2017DecoupledWD} z wykładniczym spadkiem współczynnika uczenia oraz zanikiem wag (ang. \textit{weight decay}).
                
%                     \paragraph{Parametry}
    
%                          Parametry optymalizatora: $\lambda = 10^{-7}$, $\gamma = 0.99999$, zanik wag $= 10^{-7}$.  Parametry funkcji straty: $m_{pos} = 1.0$, $m_{neg} = 0.25$. Parametr warstw wyłączenia (ang. \textit{dropout}) $= 0.5$ dla obu warstw. Liczbę epok ustawiono na $5$, ze względu na ryzyko przeuczenia.
    
    
%                     \paragraph{Miara jakości}
                    
%                         Jako miarę jakości modelu wykorzystano stratę kontrastową modelu obliczoną na zbiorze walidacyjnym.

%     \subsection{Wykorzystane narzędzia}
%     - Python
%     - Rust
%     - HTML, JS 
%     - + biblioteki

% ===== ===== ===== =====
    % **TODO** [ok. 25 stron]

    % \begin{enumerate}
    %     \item Opis pipeline'u
    %     \item Wymaganie implementacyjny
    %     \item Model architektury np. C4
    %     \item Opis testów jednostkowych, integracyjnych
    %     \item Interfejs użytkownika, instrukcja użytkownika
    % \end{enumerate}
% ===== ===== ===== =====

\clearpage
\section{Projekt i Implementacja}

    % ===== ===== ===== =====
    % WYMAGANIA FUNKCJONALNE I NIEFUNKCJONALNE
    % ===== ===== ===== ===== 
    \subsection{Wymagania funkcjonalne i niefunkcjonalne}

        \subsubsection{Wymagania funkcjonalne}

            \begin{itemize}
                \item Aplikacja powinna umożliwiać przeprowadzanie klasyfikacji taksonomicznej dla wprowadzonych sekwencji
                \item Aplikacja powinna umożliwiać wybór metody grupowania sekwencji genetycznych
                \item Aplikacja powinna zawierać zaimplementowane trzy metody grupowania sekwencji genetycznych
                \item Aplikacja powinna umożliwiać porównanie jakości klasyfikacji taksonomicznej wykonanej przy użyciu różnych metod grupowania 
                \item Aplikacja powinna umożliwiać zapis wyników w formacie JSON
            \end{itemize}

        \subsubsection{Wymagania niefunkcjonalne}

            \begin{itemize}
                \item Aplikacja powinna być wykonana przy wykorzystaniu kompilowanego wysoko wydajnego języka programowania
                \item Implementacja powinna zawierać testy jednostkowe poszczególnych modułów i funkcji
            \end{itemize}

    % ===== ===== ===== =====
    % WYKORZYSTANE NARZĘDZIA
    % ===== ===== ===== ===== 
    \subsection{Wykorzystane technologie, narzędzia oraz biblioteki}

        \subsubsection{Języki programowania}

            W pracy wykorzystano języki programowania Rust\cite{Rust} oraz Python\cite{Python}.
            
            Język Python był wykorzystywany w początkowych fazach rozwoju pracy jako narzędzie do prototypowania rozwiązania oraz w ostatecznej wersji pracy do stworzenia skryptów automatyzyjących niektóre czynności związane z nauką sieci neuronowej oraz do generowania wykresów. Został on wybrany ze względu na bogatą bibliotekę standardową, dostępność wielu bibliotek zewnętrznych oraz wieloplatformowość.
            
            Język Rust został użyty do stworzenia wszystkich aplikacji oraz programów. Wybrany został ze względu na wysokość wydajność, bezpieczne zarządzanie pamięcią oraz dużą dostępność bibliotek programistycznych, które można zainstalować za pomocą menedżera pakietów \textit{cargo}\cite{Rust:cargo} dołączonego wraz ze środowiskiem języka Rust. Dodatkowymi atutami, które przyczyniły się do wyboru języka, jest bogaty system typów oraz kompilacja do kodu maszynowego. 

        \subsubsection{Biblioteki programistyczne}

            Aplikację przeglądarkową zrealizowano z wykorzystaniem biblioteki \textit{axum}\cite{Rust:axum} opartej na asynchronicznym środowisku wykonawczym \textit{tokio}\cite{Rust:tokio} języka Rust.
            Do generowania zawartości stron w formacie HTML wykorzystano silnik szablonów \textit{askama}\cite{Rust:askama}. Komunikację z bazą danych zapewniła biblioteka \textit{sqlx}\cite{Rust:sqlx}. Użyto dodatkowo biblioteki \textit{dotenv}\cite{Rust:dotenv} w celu załadowania zmiennych środowiskowych z pliku, które niezbędne są do prawidłowego działania aplikacji.

            Aplikacja konsolowa została oparta na bibliotece \textit{clap}\cite{Rust:clap}, która pozwoliła na zdefiniowanie interfejsu użytkownika, w postaci dostępnych poleceń wraz z parametrami.

            Bibioteka \textit{exquisitor-core} korzysta z biblioteki \textit{kmedoids}\cite{Schubert:2022}, która implementuje grupowanie k-medoidów oraz bibliotek pomocniczych \textit{num-traits}, \textit{tempfiles} oraz \textit{float-cmp}, które wykorzystywane są w testach jednostkowych.

            Model sieci neuronowej został zbudowany przy użyciu biblioteki \textit{burn}\cite{Rust:burn} oraz silnika obliczeniowego \textit{wgpu}.

            Ponadto w obu aplikacjach oraz bibliotece wykorzystywana jest biblioteka \textit{serde}\cite{Rust:serde} umożliwiającą serializację i deserializację danych do różnych formatów oraz biblioteka \textit{rand}\cite{Rust:rand} zapewniająca generator liczb pseudolosowych.

        \subsubsection{Narzędzia}

            W pracy zostały wykorzystane następujące narzędzia:
            \begin{itemize}
                \item \textit{cargo} jako menedżer pakietów i system budowania w Rust,
                \item \textit{rustup} do automatycznego zarządzania wersjami Rust,
                \item \textit{clippy} do statycznej analizy kodu w Rust,
                \item \textit{rustfmt} do automatycznego formatowania kodu źródłowego w Rust,
                \item \textit{cargo test} do przeprowadzania testów jednostkowych,
                \item \textit{git} jako system kontroli wersji, umożliwiający śledzenie zmian oraz zarządzanie historią kodu.
            \end{itemize}

    % ===== ===== ===== =====
    % OPIS ROZWIĄZANIA
    % ===== ===== ===== ===== 
    \subsection{Opis rozwiązania}


        \subsubsection{Potok analizy}
    % ===== ===== ===== =====
    % OPIS POTOKU
    % ===== ===== ===== =====

    % ===== ===== ===== =====
    % MODEL ARCHITEKTURY C4
    % ===== ===== ===== =====
    \subsection{Model architektury rozwiązania}

    % ===== ===== ===== =====
    % OPIS TESTÓW JEDNOSTKOWYCH
    % ===== ===== ===== =====
    \subsection{Testy}

    % ===== ===== ===== =====
    % INTERFEJS UŻYTKOWNIKA + INSTRUKCJA UŻYTKOWNIKA
    % ===== ===== ===== =====
    \subsection{Interfejs użytkownika}

\clearpage 
\section{Eksperymenty}

    W ramach pracy przeprowadzono eksperymenty polegające na porównaniu wydajności oraz jakości zaimplementowanych metod do grupowania sekwencji genetycznych w kontekście pełnej klasyfikacji taksonomicznej. W eksperymentach jako punkt odniesienia wybrano klasyfikację taksonomiczną wszystkich sekwencji wejściowych z pominięciem etapów grupowania sekwencji genetycznych.

    % ===== ===== ===== =====
    % OPIS ŚRODOWISKA EKSPERYMENTALNEGO
    % ===== ===== ===== ===== 
    \subsection{Opis środowiska eksperymentalnego}

        Eksperymenty zostały przeprowadzone na maszynie wirtualnej wykorzystującej technologię KVM o specyfikacji podanej poniżej:

        \subsubsection{Specyfikacja techniczna}\

            \begin{itemize}
                \item {
                    \textbf{System operacyjny:} Ubuntu 22.04 LTS.
                }
                \item {
                    \textbf{Procesor:} 4 rdzenie wirtualne Intel Core i7-6850K.
                }
                \item {
                    \textbf{Pamięć RAM:} 40GB.
                }
                \item {
                    \textbf{Karta graficzna:} Nvidia GeForce GTX 1080 TI.
                }
                \item {
                    \textbf{Dysk:} dysk sieciowy 1 TB z prędkością odczytu 1 Gbps.
                }
                \item {
                    \textbf{Oprogramowanie:} pakiet \texttt{BLAST} w wersji 2.16.0 oraz sterowniki karty graficznej.
                }
            \end{itemize}

    % ===== ===== ===== =====
    % MIARA JAKOŚCI
    % ===== ===== ===== =====
    \subsection{Miara jakości}

        Jako miarę jakości klasyfikacji taksonomicznej wykorzystano indeks Jaccarda, wyrażony wzorem:

        \begin{equation}
            \text{Miara jakości} = \frac{\| R \cap E \|}{\| E \cup E \|}
        \end{equation}

        gdzie 
        \begin{align*}
            R &- \text{referencyjny zbiór organizmów,} \\
            E &- \text{zbiór organizmów otrzymanych.}
        \end{align*}

        Miara została wykorzystana do porównania jakości klasyfikacji taksonomicznej przeprowadzanej z wykorzystaniem zaimplementowanych metod względem klasyfikacji taksonomicznej bez użycia potoku przetwarzania.

        Wykorzystano również dodatkowe miary jakości, które zostały zastosowane do oceny jakości utworzonych grup przez potok przetwarzania. Pierwszą wykorzystaną miarą jest znormalizowana informacja wzajemna (ang. \textit{Normalized mutual information, NMI}) i służy do oceny jakości grup, drugą jest czułość (ang. \textit{sensitivity}), która wykorzystana jest do oceny reprezentantów grup. Miary zostały zdefiniowane odpowiednio wzorami:

        \begin{equation}
            NMI(X, Y) = \frac{I(X; Y)}{\sqrt{H(X) \cdot H(Y)}}
        \end{equation}

        gdzie,
        \begin{align*}
            I(X; Y) &= \sum_{y \in Y}{ \sum_{x \in X}{p(x, y) \log_{2}{\frac{p(x, y)}{p(x) p(y)}}}}, \\
            H(X) &= - \sum_{x \in X}{ p(x) \log_{2}{p(x)}}, \\
            I(X; Y) &- \text{informacja wzajemna między zbiorami $X$ oraz $Y$}, \\
            H(X) &- \text{entropia zbioru $X$}, \\
            H(Y) &- \text{entropia zbioru $Y$}, \\
            p(x) &- \text{prawdopobieństwo zajścia zdarzenia $x$}, \\
            p(x, y) &- \text{wspólny rozkład prawdopodobieństwa $X$ oraz $Y$ }.
        \end{align*}

        \begin{equation}
            sensitivity = \frac{\text{TP}}{
                \text{TP} + \text{TN}
            }
        \end{equation}

        gdzie,
        \begin{align*}
            TP &- \text{liczba wyników prawdziwie dodatnych,} \\
            TN &- \text{liczba wyników fałszywie ujemnych.}
        \end{align*}

        Dodatkowe miary jakości zostały wykorzystane do oceny jakości grup względem grupowania z wykorzystaniem zmodyfikowanego algorytmu Needlemana-Wunscha.

    % ===== ===== ===== =====
    % ZBIÓR DANYCH
    % ===== ===== ===== =====
    \subsection{Zbiór danych}

        \subsubsection{Opis zbioru danych}
        
            W eksperymentach wykorzystano ten sam zbiór sekwencji genetycznych, co w uczeniu modelu sztucznej sieci neuronowej. Zbiór \textit{CAMI II Toy Human Microbiome Project}\cite{Fritz2019} został wybrany, ponieważ został stworzony w celu oceny wydajności algorytmów bioinformatycznych oraz zawiera znaczną liczbę sekwencji genetycznych, co umożliwiło jego wykorzystanie go zarówno w eksperymentach, jak i w uczeniu modelu sztucznej sieci neuronowej.
            
            Na podstawie zbioru stworzono podzbiory sekwencji o rozmiarach wyrażonych wzorem $2^k$ dla $k \in [0, 15]$. Zbiory są rozłączne, a do ich budowy wykorzystano wyłącznie te sekwencje, które nie zostały użyte do uczenia modelu sztucznej sieci neuronowej.

        \subsubsection{Przygotowanie zbioru danych}

            Podzbiory powstały poprzez losowanie bez zwracania indeksów sekwencji genetycznych ze zbioru referencyjnego, które powinny trafić do podzbioru. W losowaniu wykorzystano informację o indeksach sekwencji wykorzystanych w zbiorze uczącym oraz walidacyjnym sztucznej sieci neuronowej. W celu umożliwienia dalszego wykorzystania zbioru danych zapisano do pliku indeksy sekwencji wykorzystanych w procesie budowy zbioru eksperymentalnego.

    % ===== ===== ===== =====
    % PROCEDURA PRZEPROWADZANIA EKSPERYMENTÓW
    % ===== ===== ===== =====
    \subsection{Procedura przeprowadzania eksperymentów}

        Eksperymenty zostały przeprowadzone za pomocą zaimplementowane polecenia aplikacji konsolowej, które pozwala na nadzorowanie wykorzystania procesora oraz pamięci RAM przez eksperyment, oraz kontrolowanie maksymalnego czasu trwania eksperymentu.

        Dane do obu eksperymentów zostały zgromadzone w wyniku przeprowadzenia klasyfikacji taksonomicznej z wykorzystaniem zaimplementowanych metod oraz klasyfikacji taksonomicznej bez wykorzystania potoku przetwarzania dla każdego podzbioru osobno.

        \subsubsection{Parametry}

            Maksymalny czas trwania jednej klasyfikacji taksonomicznej został ograniczony do 12 godzin. Dla wszystkich metod wybrano algorytm grupowania k-medoidów oraz określono liczbę tworzonych grup przez algorytm grupowania za pomocą wzoru $\sqrt{n}$, gdzie $n$ to liczba sekwencji wejściowych. W metodzie z wykorzystaniem zmodyfikowanego algorytmu Needlemana-Wunscha wykorzystano parametry określone w sekcji~\ref{Method:NeedlemaWunsch}. W przypadku metody wykorzystującej zanurzenia $k$-merów parametr $k$ został ustawiony na $4$. W metodzie z wykorzystaniem sztucznej sieci neuronowej wykorzystano wcześniej stworzony model.

        \subsubsection{Skrypty}

            Do przeprowadzenia eksperymentów zostały wykorzystane opracowane skrypty w języku \texttt{bash}. Każdy skrypt uruchamia klasyfikację taksonomiczną dla podzbiorów z wykorzystaniem jednej metody. Uruchomienie wszystkich skryptów pozwala na zebranie danych dla obu eksperymentów.

            \todo{
                Uruchomienie skryptów?
                Przykład skryptu?
            }
            
            Do przetworzenia wyników oraz wygenerowania wykresów wykorzystano skrypty stworzone w języku \texttt{Python}.

            \todo{
                Uruchomienie skryptu?
                Przykład skryptu?
            }

    % ===== ===== ===== =====
    % WYNIKI
    % ===== ===== ===== =====
    \subsection{Wyniki}

        \todo{
            \begin{enumerate}
                \item {opis przeprowadzonych eksperymentów}
            \end{enumerate}
        }

        \subsubsection{Eksperyment 1: czas wykonania klasyfikacji taksonomicznej}

            \todo{
                \begin{enumerate}
                    \item {cel,}
                    \item {założenia,}
                    \item {wyniki,}
                    \item {wykres/tabela,}
                    \item {interpretacja}
                \end{enumerate}
            }

        \subsubsection{Eksperyment 2: jakość klasyfikacjia taksonomicznej}

            \todo{
                \begin{enumerate}
                    \item {cel,}
                    \item {założenia,}
                    \item {wyniki,}
                    \item {wykres/tabela,}
                    \item {interpretacja}
                \end{enumerate}
            }

        \subsubsection{Analiza wyników}

            \todo{
                \begin{enumerate}
                    \item {porównanie osiągniętych wyników z oczekiwaniami,}
                    \item {dyskusja nad błędami i ich wpływem na eksperymenty}
                \end{enumerate}
            }


% \clearpage
%     \section{Eksperymenty}

%         Przeprowadzono eksperymenty jakościowe oraz wydajnościowe dla trzech metod oraz dodatkowo dla klasyfikacji taksonomicznej.

%         \subsection{Zbiory danych}
%             Do przeprowadzenia eksperymentów wykorzystano zbiór \textit{CAMI II Toy Human Microbiome Project}, z którego wylosowano zbiór 2047 sekwencji genetycznych, które nie były wykorzystane w procesie uczenia sieci neuronowej. Zbiór następnie został podzielony na podzbiory o wielkościach $n^k$ dla $k \in [0, 10]$.

%         \subsection{Miara jakości}
%             Do oceny jakości wyników pełnej klasyfikacji taksonomicznej został wykorzystany zmodyfikowany indeks Jaccarda, wyrażony wzorem:

%             \begin{equation}
%                 \text{Miara jakosci} = \frac{1}{|| R \cup E||} \sum_{o \in (R \cup E)}{
%                     \frac{\min{(R_o, E_o)}}{\max{(R_o, E_o)}}
%                 }
%             \end{equation}
    
%             gdzie 
%             \begin{align*}
%                 R, E &- \text{zbiory organizmów referencyjnych i otrzymanych}, \\
%                 E_o, R_o &- \text{jakość organizmów w zbiorach } E \text{ i } R.
%             \end{align*}

%             Wykorzystano również dodatkowe miary jakości, które oceniają tworzone klastry w wyniku wykorzystania różnych metod określania niepodobieństwa. Pierwszą z nich jest znormalizowana informacja wzajemna (ang. \text{Normalized mutual information, NMI}) i służy do oceny klastrów, drugą jest czułość wykorzystywaną do oceny reprezentantów (ang. \textit{sensitivity}). Miary zdefiniowane są jako:

%             \begin{equation}
%                 sensitivity = \frac{\text{liczba wyników prawdziwie dodatnych}}{
%                     \text{liczba wyników prawdziwie dodatnych} + \text{liczba wyników fałszywie ujemnych}
%                 }
%             \end{equation}
            
%             \begin{equation}
%                 NMI = \frac{I(X; Y)}{\sqrt{H(X) \cdot H(Y)}}
%             \end{equation}

%             gdzie 
%             \begin{align*}
%                 I(X; Y) &= \sum_{y \in Y}{ \sum_{x \in X}{p(x, y) \log_{2}{\frac{p(x, y)}{p(x) p(y)}}}}, \\
%                 H(X) &= - \sum_{x \in X}{ p(x) \log_{2}{p(x)}}, \\
%                 I(X; Y) &- \text{informacja wzajemna między zbiorami $X$ oraz $Y$}, \\
%                 H(X) &- \text{entropia zbioru $X$}, \\
%                 H(Y) &- \text{entropia zbioru $Y$}, \\
%                 p(x) &- \text{prawdopobieństwo zajścia zdarzenia $x$}, \\
%                 p(x, y) &- \text{wspólny rozkład prawdopodobieństwa $X$ oraz $Y$ }.
%             \end{align*}

%         \subsection{Procedura}

%             % \subsubsection{Wykorzystany sprzęt}
%             % Eksperymenty zostały przeprowadzone na serwerze wyposażonym w procesor Intel Core i7-6850KK (4 vCPU), 40GB pamięci RAM oraz kartę graficzną GTX 1080TI.

%             % \subsubsection{Uruchomienie}
%            Eksperymenty zostały przeprowadzone przy wykorzystaniu zbudowanego narzędzia, które monitorowało wykorzystanie procesora oraz pamięci przez eksperyment. Maksymalny czas eksperymentu ustawiono na 4 godziny. Liczbę grup ustawiono na $\sqrt{n}$, gdzie $n$ to liczba sekwencji w danym eksperymencie.

% \begin{enumerate}
%     \item Badania jakościowe
%     \begin{itemize}
%         \item opis bazy / referencji
%         \item miara jakości
%         \item 2-3 eksperymenty
%     \end{itemize}
%     \item Badania wydajnościowe
%     \begin{itemize}
%         \item pamięć, CPU
%         \item 2-3 eksperymenty  (w zależności od parametrów)
%     \end{itemize}
%     \item Wnioski
% \end{enumerate}

\cleardoublepage
\section{Podsumowanie}

    % ===== ===== ===== =====
    % GŁÓWNE CELE PRACY
    % ===== ===== ===== =====
    \subsection{Główne cele pracy}
        \todo{
            \begin{enumerate}
                \item {krótkie przepomnienie celu pracy oraz eksperymentów,}
                \item {podsumowanie problemów, które zostały rozwiązane}
            \end{enumerate}
        }

    % ===== ===== ===== =====
    % WYNIKI I OSIĄGNIĘCIA
    % ===== ===== ===== =====
    \subsection{Wyniki i osiągnięcia}
        \todo{
            \begin{enumerate}
                \item {najważniejsze cechy wytworzonego rozwiązania,}
                \item {wnioski z eksperymentów}
            \end{enumerate}
        }

    % ===== ===== ===== =====
    % OGRANICZENIA
    % ===== ===== ===== =====
    \subsection{Ograniczenia}

        \subsubsection{Ograniczona funkcjonalność narzędzi do sieci neuronowych w Rust}

            Mimo rosnącej popularności języka Rust\cite{Rust:popularity} i dynamicznego rozwoju jego ekosystemu, narzędzia do tworzenia modeli sztucznych sieci neuronowych są nadal w fazie rozwoju. Dostępne biblioteki w większości oferują interfejsy do istniejących rozwiązań w języku C++ lub w mniejszym stopniu są pisane od podstaw. Ze względu na wczesny etap rozwoju pierwszy typ tych narzędzi nie wykorzystuje w pełni możliwości języka Rust, a drugi typ nie zapewnia jeszcze pełnej optymalizacji procesu uczenia modeli sztucznych sieci neuronowych w porównaniu do bardziej dojrzałych rozwiązań dostępnych w językach C++ czy Python. Wykorzystana w pracy biblioteka \texttt{burn} wraz ze środowiskiem \texttt{wgpu} napotkała ograniczenia, które uniemożliwiły w pełni wykorzystanie potencjału dostępnej karty graficznej. Biblioteka na obecnym poziomie rozwoju zawiera jedynie implementacje podstawowych i klasycznych rozwiązań w dziedzinie sztucznych sieci neuronowych.

        \subsubsection{Złożoność procesu uczenia modelu sztucznej sieci neuronowej}

            Wykorzystanie dużego zbioru uczącego zawierającego milion przykładów w procesie uczenia znacząco spowolniło proces uczenia oraz strojenia modelu sztucznej sieci neuronowej. Ze względu na ograniczenia wykorzystanego narzędzia do tworzenia modelu wykonanie jednej epoki trwało około 1 godzinę na karcie graficznej NVIDIA RTX 2060 Super. Wybór tak dużego zbioru uczącego był podyktowany wysoką złożonością przestrzeni danych, której rozmiar w przypadku wykorzystania sekwencji DNA o długości 150 wynosi $4^{150}$, co daje w przybliżeniu $10^{90}$ różnych sekwencji DNA.

        \subsubsection{Czasochłonność eksperymentów}

            Przeprowadzenie całego procesu klasyfikacji taksonomicznej sekwencji DNA z wykorzystaniem narzędzia \texttt{BLASTn} w ramach eksperymentów dla każdej metody i każdego podzbioru eksperymentalnego było procesem czasochłonnym. Łączny czas wykonania eksperymentów opisanych w pracy wyniósł około 5 dni, co spowodowało ograniczenie liczby eksperymentów do jednego przebiegu i rozmiaru podzbioru eksperymentalnego do maksymalnie $4096$ sekwencji.

        \subsubsection{Wykorzystanie jednego zbioru danych}

            W pracy wykorzystano jedynie część dostępnego zbioru danych, ograniczając się do jednej próbki ze względu na jej duży rozmiar. Wybór próbki zawierającej sekwencje DNA pochodzące z mikrobiomu skóry człowieka mógł ograniczyć przestrzeń analizowanych sekwencji, co mogło wpłynąć na wyniki eksperymentów, ponieważ w eksperymentach również wykorzystano sekwencje DNA pochodzące z tego samego mikrobiomu.

        \subsubsection{Złożoność obliczeniowa macierzy niepodobieństwa}

            Czas budowy macierzy niepodobieństwa rośnie wprost proporcjonalnie do kwadratu liczby sekwencji, które zostały użyte do jej budowy. Macierz niepodobieństwa wykorzystywana przez algorytm grupowania w pracy była budowana dla wszystkich sekwencji wejściowych. Zastosowane podejście znacznie ogranicza liczbę możliwych sekwencji wejściowych ze względu na szybki wzrost czasu potrzebnego na tworzenie macierzy niepodobieństwa.

    % ===== ===== ===== =====
    % MOŻLIWOŚCI DALSZEGO ROZWOJU
    % ===== ===== ===== =====
    \subsection{Możliwości dalszego rozwoju}

        \subsubsection{Zmiana architektury modelu sztucznej sieci neuronowej}

            Stworzony model sztucznej sieci neuronowej wymaga sekwencji o stałej długości, co ogranicza elastyczność modelu w analizie sekwencji o różnych długościach i wymaga tworzenia nowego modelu dostosowanego do dłuższych sekwencji w przypadku znacznej różnicy między długościami sekwencji wejściowych a oczekiwanych przez model. Można rozważyć zmianę architektury modelu sztucznej sieci neuronowej, dostosowując ją do analizy sekwencji o zmiennych długościach poprzez zastosowanie sieci rekurencyjnych lub typu LSTM w pierwszych warstwach modelu.

        \subsubsection{Grupowanie sekwencji w paczkach}

            - wstępne grupowanie sekwencji w losowe paczki
            - nastepnie wybieranie z tego reprezentantów
            - ominięcie budowy dużej macierzy niepodobieństwa

        \subsubsection{Wykonywanie obliczeń równolegle}

            - wykonywanie obliczeń równolegle np. przy budowie macierzy niepodobieństwa

        \subsubsection{Automatyczny dobór liczby grup}
            
            - automatyczne dostosowanie liczby grup - nie trzeba stroić przez użytkownika

        \subsubsection{Narzędzie do klasyfikacji taksonomicznej oparte o model sztucznej sieci neuronowej}

            - stworzenie własnego narzedzia, które wykorzystuje zanurzenia bezpośrednio do klasyfikacji taksonomicznej

        \todo{
            \begin{enumerate}
                \item {propozycje rozwoju systemu,}
                \item {wskazanie obszarów, które mogą zostać poprawione}
            \end{enumerate}
        }


    % ===== ===== ===== =====
    % WNIOSKI KOŃCOWE
    % ===== ===== ===== =====
    \subsection{Wnioski końcowe}

        \todo{
            \begin{enumerate}
                \item {podsumowanie głównych wniosków z pracy,}
                \item {podkręslenie najważniejszych osiągnięć}
            \end{enumerate}
        }

% ---- ---- ---- ----
% Bibliography
% ---- ---- ---- ----
\cleardoublepage % Zaczynamy od nieparzystej strony
\printbibliography
\clearpage

% ---- ---- ---- ----
% Wykaz symboli i skrótów
% ---- ---- ---- ----
\acronymlist
\acronym{BZ}{Klasyfikacja taksonomiczna bez wykorzystania potoku przetwarzania}
\acronym{NW}{Metoda z wykorzystaniem zmodyfikowanego algorytmu Needlamana-Wunscha}
\acronym{$k$-mer}{Metoda z wykorzystanie zanurzeń $k$-merów}
\acronym{SSN}{Metoda z wykorzystaniem sztucznej sieci neuronowej}
\vspace{1cm}

% ---- ---- ---- ----
% Spis rysunków
% ---- ---- ---- ----
\pagestyle{plain}
\listoffigurestoc{}
\vspace{1cm}

% ---- ---- ---- ----
% Spis tabel
% ---- ---- ---- ----
\listoftablestoc{}
\vspace{1cm}

% ---- ---- ---- ----
% Spis załączników
% ---- ---- ---- ----
\listofappendicestoc

% ---- ---- ---- ----
% Załączniki
% ---- ---- ---- ----
\clearpage
\appendix{Kod źródłowy}
Kod rozwiązania wytworzony w ramach pracy dostępny jest w dołączonym archiwum ZIP o nazwie \texttt{exquisitor.zip}.


\end{document}
